\let\negmedspace\undefined
\let\negthickspace\undefined
\documentclass[journal]{IEEEtran}
\usepackage[a5paper, margin=10mm, onecolumn]{geometry}
\usepackage{lmodern} % Ensure lmodern is loaded for pdflatex
\usepackage{tfrupee} % Include tfrupee package

\setlength{\headheight}{1cm} % Set the height of the header box
\setlength{\headsep}{0mm}     % Set the distance between the header box and the top of the text

\usepackage{gvv-book}
\usepackage{gvv}
\usepackage{cite}
\usepackage{amsmath,amssymb,amsfonts,amsthm}
\usepackage{algorithmic}
\usepackage{graphicx}
\usepackage{textcomp}
\usepackage{xcolor}
\usepackage{txfonts}
\usepackage{listings}
\usepackage{enumitem}
\usepackage{mathtools}
\usepackage{gensymb}
\usepackage{comment}
\usepackage[breaklinks=true]{hyperref}
\usepackage{tkz-euclide} 
\usepackage{listings}
% \usepackage{gvv}                                        
\def\inputGnumericTable{}                                 
\usepackage[latin1]{inputenc}                                
\usepackage{color}                                            
\usepackage{array}                                            
\usepackage{longtable}                                       
\usepackage{calc}                                             
\usepackage{multirow}                                         
\usepackage{hhline}                                           
\usepackage{ifthen}                                           
\usepackage{lscape}
\begin{document}

\bibliographystyle{IEEEtran}
\vspace{3cm}

\title{2021-PE-27-39}
\author{EE24BTECH11010 - BALAJI B}
% \maketitle
% \newpage
% \bigskip
{\let\newpage\relax\maketitle}

\renewcommand{\thefigure}{\theenumi}
\renewcommand{\thetable}{\theenumi}
\setlength{\intextsep}{10pt} % Space between text and floats


\numberwithin{equation}{enumi}
\numberwithin{figure}{enumi}
\renewcommand{\thetable}{\theenumi}
\begin{enumerate}
    \item The donor concentration in a sample of $n$-type silicon is increased by a factor
of 100. Assuming the sample to be non-degenerate, the shift in the Fermi
level (in $meV$) at $300 K$ (rounded off to the nearest integer) is \rule{2cm}{0.4pt}\\ (Given $k_BT = 25 meV$ at $300K$)\hfill [2021 PH]
\item Two observers $O$ and $O^\prime$ observe two events $P$ and $Q$. The observers have a
constant relative speed of $0.5c$. In the units, where the speed of light, $c$, is
taken as unity, the observer $O$ obtained the following coordinates: \\
Event $P$ : $x = 5$, $y = 3$, $z = 5$, $t = 3$ \\
Event $Q$ : $x = 5$, $y = 1$, $z = 3$, $t = 5$\\
The length of the space-time interval between these two events, as measured by $O^\prime$, is $L$. The value of $\abs{L}$ (in integer) is \rule{2cm}{0.4pt}\hfill [2021 PH]
\item A light source having its intensity peak at the wavelength $289.8 nm$ is
calibrated as $10,000 K$ which is the temperature of an equivalent black body
radiation. Considering the same calibration, the temperature of light source
(in $K$) having its intensity peak at the wavelength $579.6 nm$ (rounded off to
the nearest integer) is \rule{2cm}{0.4pt}

\hfill [2021 PH]
\item A hoop of mass $M$ and radius $R$ rolls without slipping along a straight line
on a horizontal surface as shown in the figure. A point mass $m$ slides without
friction along the inner surface of the hoop, performing small oscillations
about the mean position. The number of degrees of freedom of the system
(in integer) is \rule{1.5cm}{0.4pt}

\hfill [2021 PH]
\begin{figure}[H]
\centering
\resizebox{10cm}{!}{%
\begin{circuitikz}
\tikzstyle{every node}=[font=\normalsize]
\draw (6,14) to[R] (8,14);
\draw (4.5,14) to[L ] (6.5,14);
\draw (8,14) to[short] (9,14);
\draw (9,14) to[short] (9,13.5);
\draw (9,13.5) to[european resistor] (9,12);
\draw (9,12) to[short] (4.5,12);
\draw (4.5,14.5) to[short] (4.5,11.75);
\draw (4.5,14.5) to[short] (1.75,14.5);
\draw (1.75,14.5) to[short] (1.75,11.75);
\draw (1.75,11.75) to[short] (4.5,11.75);
\draw [short] (3,14) -- (3,13.5);
\draw [short] (2.5,13.5) -- (3.5,13.5);
\draw [short] (3,13.5) -- (2.5,13);
\draw [short] (3,13.5) -- (3.5,13);
\draw [short] (2.5,13) -- (3.5,13);
\draw [short] (3,13) -- (3,12.5);
\draw [short] (3.25,13.5) -- (3.5,13.75);
\node [font=\normalsize] at (5.5,14.75) {Filter};
\node [font=\normalsize] at (7,14.75) {Choke};
\node [font=\normalsize] at (10,12.75) {Battery};
\draw [short] (-2,14.75) -- (-2,11.75);
\draw [short] (-2.5,14.75) -- (-2.5,11.75);
\draw (-3.25,13.25) to[tmultiwire] (-6.75,13.25);
\draw (1.75,13.25) to[tmultiwire] (-1.25,13.25);
\draw  (-7.5,13.25) circle (0.75cm);
\draw [short] (-8,13.25) .. controls (-7.5,13.75) and (-7.5,12.5) .. (-7,13.25);
\draw [decorate, decoration={coil, aspect=0.4, segment length=6pt, amplitude=10pt}] (-1.5,12) -- (-1.5,14.5);
\draw [decorate, decoration={coil, aspect=0.4, segment length=6pt, amplitude=10pt}] (-3,12) -- (-3,14.5);
\end{circuitikz}
}%

\label{fig:my_label}
\end{figure}
\item Three non-interacting bosonic particles of mass $m$ each, are in a one-
dimensional infinite potential well of width $a$. The energy of the third excited
state of the system is $x \times \frac{h^2 \pi^2}{ma^2}$. The value of $x$ (in integer) is \rule{2cm}{0.4pt} 

\hfill [2021 PH]
\item The spacing between two consecutive S- 
branch lines of the rotational Raman
spectra of hydrogen gas is $243. 2 cm^{-1}$
. After excitation with a laser of
wavelength $514.5 cm$, the Stoke's line appeared at $17611.4 cm^{-1}$
for a
particular energy level. The wavenumber (rounded off to the nearest
integer), in $cm^{-1}$
, at which Stoke's line will appear for the next higher energy
level is \rule{2cm}{0.4pt}\hfill [2021 PH]
\item The transition line, as shown in the figure, arises between $^2D_{\frac{3}{2}}$ and $^2P_\frac{1}{2}$ states without any external magnetic field. The number of
lines that will appear in the presence of a weak magnetic field (in integer) is \rule{2cm}{0.4pt} \hfill [2021 PH] 
\begin{figure}[!ht]
\centering
\resizebox{6cm}{!}{%
\begin{circuitikz}
\tikzstyle{every node}=[font=\normalsize]
\draw [line width=0.5pt, short] (3,12.5) .. controls (2.5,14.5) and (3,14.75) .. (4.25,14.25);
\draw [line width=0.5pt, short] (2.75,12.5) .. controls (2.25,13.75) and (2.25,15.25) .. (4,14.75);
\draw [line width=0.5pt, short] (4,14.75) -- (6.5,14);
\draw [line width=0.5pt, short] (4.25,14.25) -- (6.5,13.5);
\draw [line width=0.5pt, short] (6.5,13.5) .. controls (8.25,13.5) and (8.75,13.75) .. (8.75,14.5);
\draw [line width=0.5pt, short] (6.5,14) .. controls (8,13.75) and (8.5,14) .. (8.5,14.75);
\draw [line width=0.5pt, short] (8.75,14.5) .. controls (9,15.25) and (8.75,15.5) .. (8,16.25);
\draw [line width=0.5pt, short] (8.5,14.75) .. controls (8.25,15.5) and (8.25,15.75) .. (7.5,16);
\draw [line width=0.5pt, short] (7.5,16) .. controls (7.75,16) and (8,16) .. (8,16.25);
\draw [line width=0.5pt, short] (7.5,16) .. controls (7.75,16.25) and (7.75,16.25) .. (8,16.25);
\draw [line width=0.5pt, ->, >=Stealth, dashed] (8,16.25) -- (8.5,18);
\draw [line width=0.5pt, ->, >=Stealth, dashed] (8,16.25) .. controls (7,17) and (6.5,17) .. (6.25,18.25);
\draw [line width=0.5pt, ->, >=Stealth, dashed] (7.75,16.25) -- (5.25,17);
\draw [line width=0.5pt, ->, >=Stealth, dashed] (7.75,16) .. controls (6.25,16.25) and (6.25,16.25) .. (5.25,15.75);
\node [font=\normalsize] at (8.25,16.25) {$E$};
\node [font=\normalsize] at (6.75,18) {$p$};
\node [font=\normalsize] at (8.75,18) {$q$};
\node [font=\normalsize] at (6,17.25) {$n$};
\node [font=\normalsize] at (6,15.75) {$m$};
\end{circuitikz}
}%

\label{fig:my_label}
\end{figure}

\item Consider the atomic system as shown in the figure, where the Einstein $A$
coefficients for spontaneous emission for the levels are $A_{2 \to 1} = 2 \times 10^7 s^{-1}$ and $A_{1 \to 0} = 10^8 s^{-1}$. If $10^{14}$ atoms/$cm^3$ are excited from level 0 to level 2 and a steady state population in level 2 is achieved, then the steady state
population at level 1 will be $x \times 10^{13} $
. The value of $x$ (in integer) is \rule{2cm}{0.4pt}\hfill [2021 PH] 
\begin{figure}[!ht]
\centering
\resizebox{6.5cm}{!}{%
\begin{circuitikz}
\tikzstyle{every node}=[font=\large]
\draw [line width=0.7pt, short] (3.25,13.25) -- (10.25,11.5);
\draw [line width=0.7pt, short] (3.25,13.25) -- (3.5,13.5);
\draw [line width=0.7pt, short] (3.5,13.5) -- (10.5,11.75);
\draw [line width=0.7pt, short] (10.5,11.75) -- (10.25,11.5);
\draw [line width=0.7pt, ->, >=Stealth] (3.5,13.5) -- (4.5,14.25);
\draw [line width=0.7pt, ->, >=Stealth] (10.5,11.75) -- (11.75,13);
\node [font=\large] at (3.25,12.75) {$E$};
\node [font=\large] at (10,11.25) {$F$};
\node [font=\large] at (4,14.25) {$V_E$};
\node [font=\large] at (12,12.5) {$V_F$};
\end{circuitikz}
}%

\label{fig:my_label}
\end{figure}
\item If $\vec{a}$ and $\vec{b}$ are constant vectors, $\vec{r}$ and $\vec{p}$ are generalized positions and
conjugate momenta, respectively, then for the transformation $Q = \vec{a}\cdot \vec{p}$ and
$P = \vec{b} \cdot \vec{r}$ to be canonical, the value of $\vec{a} \cdot \vec{b}$ (in integer) is \rule{2cm}{0.4pt}\hfill [2021 PH]

\item The below combination of logic gates represents the operation \\
\begin{figure}[!ht]
\centering
\resizebox{7cm}{!}{%
\begin{circuitikz}
\tikzstyle{every node}=[font=\large]
\draw [ line width=0.7pt](-3,15.75) to[american voltage source] (-3,13);
\draw [ line width=0.7pt](-3,15.75) to[short] (-1.5,15.75);
\draw [ line width=0.7pt](-3,13) to[short] (-1.5,13);
\draw [ line width=0.7pt](-1.5,16.5) to[short] (-1.5,12.25);
\draw [ line width=0.7pt](-1.5,16.5) to[short] (2.25,16.5);
\draw [ line width=0.7pt](2.25,16.5) to[short] (2.25,12.25);
\draw [ line width=0.7pt](-1.5,12.25) to[short] (2.25,12.25);
\draw [ line width=0.7pt](2.25,13.25) to[short] (5.25,13.25);
\draw [line width=0.7pt](5.25,13.25) to[C] (5.25,15.75);
\draw [line width=0.7pt](2.25,15.75) to[L ] (5.25,15.75);
\draw [ line width=0.7pt](5.25,15.75) to[short] (6.75,15.75);
\draw [ line width=0.7pt](5.25,13.25) to[short] (6.75,13.25);
\draw [ line width=0.7pt](6.75,15.75) to[R] (6.75,13.25);
\node [font=\large] at (-3.75,14.25) {$V_{in}$};
\node [font=\large] at (0.5,14.5) {Full-Bridge};
\node [font=\large] at (0.25,13.75) {VSI};
\node [font=\large] at (2.5,14.5) {$V_R$};
\node [font=\large] at (2.5,16) {$+$};
\node [font=\LARGE] at (2.5,13) {$-$};
\node [font=\large] at (4.5,14.5) {$C$};
\node [font=\large] at (6.25,14.5) {$R$};
\node [font=\large] at (7.25,14.5) {$V_o$};
\node [font=\large] at (7,15.5) {$+$};
\node [font=\LARGE] at (7,13.25) {$-$};
\node [font=\large] at (3.75,16.5) {$L$};
\end{circuitikz}
}%

\label{fig:my_label}
\end{figure}
 \hfill[2021 PH]
\item In a semiconductor, the ratio of the effective mass of hole to electron is $2:11$
and the ratio of average relaxation time for hole to electron is $1:2$. The ratio
of the mobility of the hole to electron is \hfill [2021 PH]
\begin{enumerate}
    \begin{multicols}{2}
        \item $4:9$
        \item $4:11$
        \item $9:4$
        \item $11:4$
    \end{multicols}
\end{enumerate}
\item Consider a spin $S = \frac{\hbar}{2}$ particle in the state $| \phi \rangle = \frac{1}{3} \myvec{2 + i \\ 2}$. The probability that a measurement finds the state with $S_x = + \frac{\hbar}{2}$ is  \hfill [2021 PH]
\begin{enumerate}
    \begin{multicols}{4}
        \item $\frac{5}{18}$
         \item $\frac{11}{18}$
          \item $\frac{15}{18}$
           \item $\frac{17}{18}$
    \end{multicols}
\end{enumerate}
\item An electromagnetic wave having electric field $E = 8 \cos\brak{{kz - \omega t}} \hat{y}V cm^{-1}$ is incident at $90 \degree$(normal incidence) on a square slab from vacuum (with
refractive index $n_0 = 1.0$) as shown in the figure. The slab is composed of
two different materials with refractive indices $n_1$ and $n_2$. Assume that the
permeability of each medium is the same. After passing through the slab for
the first time, the electric field amplitude, in $V cm^{-1}$, of of the electromagnetic
wave, which emerges from the slab in region 2, is closest to \hfill [2021 PH]
\begin{figure}[H]
\centering
\resizebox{2.5cm}{!}{%
\begin{circuitikz}
\tikzstyle{every node}=[font=\Large]
\node [font=\Large, rotate around={180:(0,0)}] at (1.75,14.25) {T};
\node [font=\Large, rotate around={180:(0,0)}] at (2.15,14.25) {R};
\node [font=\Large] at (2.5,14.25) {I};
\node [font=\Large, rotate around={180:(0,0)}] at (2.75,14.25) {A};
\node [font=\Large, rotate around={-180:(0,0)}] at (3.25,14.25) {N};
\node [scale=-1, yscale=-1, font=\Large, rotate around={180:(0,0)}] at (3.75,14.25) {G};
\node [font=\Large, rotate around={180:(0,0)}] at (4.15,14.25) {L};
\node [font=\Large] at (4.5,14.25) {E};
\end{circuitikz}
}%

\label{fig:my_label}
\end{figure}

\begin{enumerate}
    \begin{multicols}{4}
        \item $\frac{11}{1.6}$
        \item $\frac{11}{3.2}$
        \item $\frac{11}{13.8}$
        \item $\frac{11}{25.6}$
    \end{multicols}
\end{enumerate}

\end{enumerate}
\end{document}