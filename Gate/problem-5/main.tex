\let\negmedspace\undefined
\let\negthickspace\undefined
\documentclass[journal]{IEEEtran}
\usepackage[a5paper, margin=10mm, onecolumn]{geometry}
\usepackage{lmodern} % Ensure lmodern is loaded for pdflatex
\usepackage{tfrupee} % Include tfrupee package

\setlength{\headheight}{1cm} % Set the height of the header box
\setlength{\headsep}{0mm}     % Set the distance between the header box and the top of the text

\usepackage{gvv-book}
\usepackage{gvv}
\usepackage{cite}
\usepackage{amsmath,amssymb,amsfonts,amsthm}
\usepackage{algorithmic}
\usepackage{graphicx}
\usepackage{textcomp}
\usepackage{xcolor}
\usepackage{txfonts}
\usepackage{listings}
\usepackage{enumitem}
\usepackage{mathtools}
\usepackage{gensymb}
\usepackage{comment}
\usepackage[breaklinks=true]{hyperref}
\usepackage{tkz-euclide} 
\usepackage{listings}
% \usepackage{gvv}                                        
\def\inputGnumericTable{}                                 
\usepackage[latin1]{inputenc}                                
\usepackage{color}                                            
\usepackage{array}                                            
\usepackage{longtable}                                       
\usepackage{calc}                                             
\usepackage{multirow}                                         
\usepackage{hhline}                                           
\usepackage{ifthen}                                           
\usepackage{lscape}
\begin{document}

\bibliographystyle{IEEEtran}
\vspace{3cm}

\title{2015-EE-40-52}
\author{EE24BTECH11010 - BALAJI B}
% \maketitle
% \newpage
% \bigskip
{\let\newpage\relax\maketitle}

\renewcommand{\thefigure}{\theenumi}
\renewcommand{\thetable}{\theenumi}
\setlength{\intextsep}{10pt} % Space between text and floats


\numberwithin{equation}{enumi}
\numberwithin{figure}{enumi}
\renewcommand{\thetable}{\theenumi}
\begin{enumerate}
    \item An unbalanced $DC$
 Wheatstone bridge is shown in the figure. At what value of $p$
 will the magnitude of $V_0$
 be maximum? \hfill(2015-EE)
 \begin{figure}[H]
\centering
\resizebox{3cm}{!}{%
\begin{circuitikz}
\tikzstyle{every node}=[font=\normalsize]
\draw [line width=0.5pt, ->, >=Stealth] (1.5,11.5) -- (1.5,15);
\draw [line width=0.5pt, ->, >=Stealth] (1.5,11.5) -- (6,11.5);
\node [font=\Huge] at (2.5,13.25) {P};
\node [font=\Huge] at (3,13.25) {H};
\node [font=\Huge] at (3.5,13.25) {Y};
\node [font=\Huge] at (4,13.25) {L};
\node [font=\Huge] at (4.5,13.25) {A};
\node [font=\Huge] at (5,13.25) {X};
\node [font=\Huge] at (5.5,13.25) {I};
\node [font=\Huge] at (6,13.25) {S};
\node [font=\normalsize] at (1.25,14.75) {Y};
\node [font=\normalsize] at (5.75,11) {X};
\end{circuitikz}
}%

\label{fig:my_label}
\end{figure}

 \begin{enumerate}
     \begin{multicols}{4}
         \item $\sqrt{\brak{1+x}}$
         \item $\brak{1+x}$
         \item $\frac{1}{\sqrt{\brak{1+x}}}$
         \item $\sqrt{\brak{1-x}}$
     \end{multicols}
 \end{enumerate}
 \item The circuit shown is meant to supply a resistive load $R_L$
 from two separate $DC$
 voltage sources. The switches $S_1$
 and $S_2$
 are controlled so that only one of them is ON at any instant. $S_1$
 is turned on for $0.2ms$
 and $S_2$
 is turned on for $0.3ms$
 in a $0.5ms$
 switching cycle time period. Assuming continuous conduction of the inductor current and negligible ripple on the capacitor voltage, the output voltage $V_o$
 (in Volt) across $R_L$
 is \rule{2cm}{0.4pt}. \hfill(2015-EE)
 \begin{figure}[H]
\centering
\resizebox{6cm}{!}{%
\begin{circuitikz}
\tikzstyle{every node}=[font=\large]
\draw (0.25,17.75) to[short, -o] (-1,17.75) ;
\draw (-0.25,17.75) to[curved capacitor] (1.5,17.75);
\draw (1.5,17.75) to[short, -o] (4.75,17.75) ;
\draw (3.25,17.75) to[R] (3.25,15.5);
\draw (3.25,15.5) to[curved capacitor] (3.25,13.75);
\draw (3.25,13.75) to[short, -o] (4.5,13.75) ;
\draw (3.25,13.75) to[short, -o] (-1.25,13.75) ;
\node [font=\large] at (-1.25,15.75) {$V_1(s)$};
\node [font=\large] at (4,16.75) {$10k\ohm$};
\node [font=\large] at (4.5,14.75) {$100\mu F$};
\node [font=\large] at (0.5,18.5) {$100\mu F$};
\node [font=\large] at (-1,18) {$+$};
\node [font=\large] at (4.75,18.25) {$+$};
\node [font=\large] at (4.5,14) {$-$};
\node [font=\large] at (-1.25,14.25) {$-$};
\node [font=\large] at (5,15.75) {$V_2(s)$};
\end{circuitikz}
}%

\label{fig:my_label}
\end{figure}

 \item A self commutating switch $SW$ 
 operated at duty cycle $\delta$
 is used to control the load voltage as shown in the figure
 \begin{figure}[H]
\centering
\resizebox{3cm}{!}{%
\begin{circuitikz}
\tikzstyle{every node}=[font=\Large]
\draw [line width=1pt, short] (-2,16.25) -- (-2,9.75);
\draw [line width=1pt, short] (-2,9.75) -- (2.25,9.75);
\draw [line width=1pt, short] (2.25,9.75) -- (2.25,16.25);
\draw [ line width=1pt ] (1,12.75) circle (1.25cm);
\draw [ line width=1pt ] (-0.75,11) circle (1.25cm);
\draw [line width=1pt, <->, >=Stealth] (-2,15.5) -- (2.25,15.5)node[pos=0.5, fill=white]{400};
\draw [line width=1pt, ->, >=Stealth] (1,12.75) -- (2,13.5);
\draw [line width=1pt, ->, >=Stealth] (-0.75,11) -- (0.25,10.25);
\node at (-0.75,11) [circ] {};
\node at (1,12.75) [circ] {};
\node at (-0.75,11) [circ] {};
\node [font=\Large] at (1,13.5) {125};
\node [font=\Large] at (0,11) {125};
\draw [line width=1.5pt, short] (2.25,16.25) -- (2.5,16.25);
\draw [line width=1.5pt, short] (2.5,16.25) -- (2.25,16.25);
\draw [line width=1.5pt, short] (-2,16.25) -- (-2.25,16.25);
\draw [line width=0.9pt, short] (2.5,16.25) -- (2.5,9.5);
\draw [line width=0.9pt, short] (2.5,9.5) -- (-2.25,9.5);
\draw [line width=0.9pt, short] (-2.25,16.25) -- (-2.25,9.5);
\node [font=\Large] at (0,8.5) {All dimensions are in $mm$};
\end{circuitikz}
}%

\label{fig:my_label}
\end{figure}
 Under steady state operating conditions, the average voltage across the indicator and the capacitor respectively, are \hfill(2015-EE)
 \begin{enumerate}
     \begin{multicols}{2}
         \item $V_L = 0$ and $V_C = \frac{1}{1-\delta} V_{dc}$
          \item $V_L = \frac{\delta}{2}V_{dc}$ and $V_C = \frac{1}{1-\delta} V_{dc}$
           \item $V_L = 0$ and $V_C = \frac{\delta}{1-\delta} V_{dc}$
            \item $V_L = \frac{\delta}{2} V_{dc}$ and $V_C = \frac{\delta}{1-\delta} V_{dc}$
     \end{multicols}
 \end{enumerate}
 \item The single-phase full-bridge voltage source inverter $\brak{VSI}$
 shown in figure, has an output frequency of $50Hz$
. It uses unipolar pulse width modulation with switching frequency of $50kHz$
 and modulation index of $0.7$
 For $V_{in} = 100V$, $L = 9.55mH$, $C = 63.66\mu F$
 and $R = 5 \ohm$
 the amplitude of the fundamental component in the voltage $V_o$ (in Volt) under steady-state is \rule{2cm}{0.4pt} \hfill(2015-EE)
 \begin{figure}[H]
\centering
\resizebox{4cm}{!}{%
\begin{circuitikz}
\tikzstyle{every node}=[font=\LARGE]
\draw (1.75,15) node[ieeestd not port, anchor=in](port){} (port.out) to[short] (3.75,15);
\draw (port.in) to[short] (1.25,15);
\draw (1.75,13) node[ieeestd not port, anchor=in](port){} (port.out) to[short] (3.75,13);
\draw (port.in) to[short] (1.25,13);
\draw [line width=1.1pt, short] (3.75,15) -- (3.75,14.25);
\draw [line width=1.1pt, short] (3.75,13) -- (3.75,13.75);
\draw [line width=1.1pt, short] (3.75,13.75) -- (4.75,13.75);
\draw [line width=1.1pt, short] (3.75,14.25) -- (4.75,14.25);
\draw (4.75,14.25) to[short] (5,14.25);
\draw (4.75,13.75) to[short] (5,13.75);
\draw (5,14.25) node[ieeestd or port, anchor=in 1, scale=0.89](port){} (port.out) to[short] (7,14);
\end{circuitikz}
}%

\label{fig:my_label}
\end{figure}
 \item A 3-phase 50 Hz square wave (6-step) $VSI$ feeds a 3-phase, 4 pole induction motor. The $VSI$ line voltage has a dominant $5^{th}$ harmonic component. If the operating slip of the motor with respect to fundamental component voltage is 0.04, the slip of the motor with respect to $5^{th}$ harmonic component of voltage is \rule{1cm}{0.4pt} 
 
 \hfill(2015-EE)
 \item A parallel plate capacitor is partially filled with glass of dielectric constant 4.0 as shown below. The dielectric strengths of air and glass are $30 kV/cm$ and $300 kV/cm$, respectively. The maximum voltage (in kilovolts), which can be applied across the capacitor without any breakdown, is \rule{2cm}{0.4pt} \hfill(2015-EE)
 \begin{figure}[H]
\centering
\resizebox{2.5cm}{!}{%
\begin{circuitikz}
\tikzstyle{every node}=[font=\LARGE]
\draw [line width=0.6pt, ->, >=Stealth] (1.25,12.5) -- (1.25,17);
\draw [line width=0.6pt, ->, >=Stealth] (1.25,12.5) -- (6.25,12.5);
\node [font=\large] at (1,16.75) {Y};
\node [font=\large] at (6,12) {X};
\node [font=\LARGE] at (2.75,15) {H};
\node [font=\LARGE] at (3.25,15) {Y};
\node [font=\LARGE, rotate around={-175:(0,0)}] at (4.25,15) {A};
\node [font=\LARGE] at (4.75,15) {X};
\node [font=\LARGE] at (5,15) {I};
\node [font=\LARGE, rotate around={180:(0,0)}] at (5.25,15) {S};
\node [font=\LARGE, rotate around={-180:(0,0)}] at (3.75,15) {L};
\node [font=\LARGE, rotate around={-180:(0,0)}] at (2.25,15) {P};
\end{circuitikz}
}%

\label{fig:my_label}
\end{figure}

 \item The figure shows a digital circuit constructed using negative edge triggered $J - K$
 flip flops. Assume a starting state of $Q_2Q_1Q_0 = 000$
 This state $Q_2Q_1Q_0 = 000$
 will repeat after \rule{1.5cm}{0.4pt} number of cycles of the clock $CLK$.
 \hfill (2015-EE)
 \begin{figure}[H]
\centering
\resizebox{2.5cm}{!}{%
\begin{circuitikz}
\tikzstyle{every node}=[font=\LARGE]
\draw [line width=0.6pt, ->, >=Stealth] (1.25,12.5) -- (1.25,17);
\draw [line width=0.6pt, ->, >=Stealth] (1.25,12.5) -- (6.25,12.5);
\node [font=\large] at (1,16.75) {Y};
\node [font=\large] at (6,12) {X};
\node [font=\LARGE] at (2.75,15) {H};
\node [font=\LARGE] at (3.25,15) {Y};
\node [font=\LARGE, rotate around={-175:(0,0)}] at (4.25,15) {A};
\node [font=\LARGE] at (4.75,15) {X};
\node [font=\LARGE] at (5,15) {I};
\node [font=\LARGE] at (5.25,15) {S};
\node [font=\LARGE, rotate around={-182:(0,0)}] at (3.75,15) {L};
\node [font=\LARGE, rotate around={-175:(0,0)}] at (2.25,15) {P};
\end{circuitikz}
}%

\label{fig:my_label}
\end{figure}

 \item $f\brak{A,B,C,D} = \Pi M \brak{0,1,3,4,5,7,9,11,12,13,14,15}$ is a maxterm representation of the Boolean function $f\brak{A,B,C,D}$ where $A$ is the MSB and $D$ is the LSB. The equivalent minimized representation of this function is \hfill(2015-EE)
 \begin{enumerate}
     \item $\brak{A + \Bar{C} + D}\brak{\Bar{A} + B + D}$
     \item $A\Bar{C}D + \Bar{A} + B + D$
     \item $\Bar{A}CD + A \Bar{B}CD + A\Bar{B}\Bar{C}\Bar{D}$
     \item $\brak{B + \Bar{C} + D}\brak{A + \Bar{B} + \Bar{C} + D} \brak{\Bar{A} + B + C + D}$
 \end{enumerate}
 \item Consider a discrete time signal given by 
 \begin{center}
     $x\sbrak{n} = \brak{-0.25}^n u\sbrak{n} + \brak{0.5}^n u \sbrak{-n-1}$
 \end{center}
 The region of convergence of its Z-transform would be \hfill (2015-EE)
 \begin{enumerate}
     \item the region inside the circle of radius 0.5 and centered at origin 
     \item the region outside the circle of radius 0.25 and centered at origin 
     \item the annular region between the two circles, both centered at origin and having radii 0.25 and 0.5
     \item the entire Z plane
 \end{enumerate}
 \item The op-amp shown in the figure has a finite gain $A= 1000$
 and an infinite input resistance. A step voltage $V_i = 1mV$
 is applied at the input at time $t = 0$
 as shown. Assuming that the operational amplifier is not saturated, the time constant (in millisecond) of the output voltage $V_o$
 is \hfill(2015-EE)
 \begin{figure}[H]
\centering
\resizebox{10cm}{!}{%
\begin{circuitikz}
\tikzstyle{every node}=[font=\small]

\draw [short] (1,17.25) -- (1,16.5);
\draw [short] (1,17.25) -- (2.25,17.25);
\draw [short] (2.25,17.25) -- (2.25,16.5);
\draw [short] (1,16.5) -- (2.25,16.5);
\draw [short] (3.5,17.25) -- (3.5,16.5);
\draw [short] (3.5,17.25) -- (4.5,17.25);
\draw [short] (4.5,17.25) -- (4.5,16.5);
\draw [short] (3.5,16.5) -- (4.5,16.5);
\draw [short] (5.25,17.25) -- (5.25,16.5);
\draw [short] (5.25,17.25) -- (6.25,17.25);
\draw [short] (6.25,17.25) -- (6.25,16.5);
\draw [short] (5.25,16.5) -- (6.25,16.5);
\draw [short] (4.5,17) -- (5.25,17);
\draw [short] (4.5,16.75) -- (5.25,16.75);
\draw [short] (6.25,17) -- (8.25,17);
\draw [short] (6.25,16.75) -- (8.25,16.75);
\draw [short] (8.25,17.25) -- (8.25,16.5);
\draw [short] (8.25,17.25) -- (9.5,17.25);
\draw [short] (9.5,17.25) -- (9.5,16.5);
\draw [short] (8.25,16.5) -- (9.5,16.5);
\draw [short] (7.25,17) -- (7.25,16.75);
\draw [short] (7.5,17) -- (7.25,16.75);
\draw [short] (7.75,17) -- (7.5,16.75);
\draw [short] (8,17) -- (7.75,16.75);
\draw [short] (8.25,17) -- (8,16.75);
\draw [short] (7,17.25) -- (10.75,17.25);
\draw [short] (7,17.25) -- (7,17.75);
\draw [short] (7,17.75) -- (10.75,17.75);
\draw [short] (10.75,17.75) -- (10.75,17.25);
\draw [short] (9.5,17) -- (10.75,17);
\draw [short] (9.5,16.75) -- (10.75,16.75);
\draw [short] (10.75,17) -- (10.75,16.75);
\draw [short] (9.75,17) -- (9.5,16.75);
\draw [short] (10,17) -- (9.75,16.75);
\draw [short] (10.25,17) -- (10,16.75);
\draw [short] (10.5,17) -- (10.25,16.75);
\draw [short] (10.75,17) -- (10.5,16.75);
\draw [->, >=Stealth] (2.25,17) -- (3.5,17);
\draw [->, >=Stealth] (10.25,16.25) -- (9.75,16.75);
\node [font=\small] at (1.5,17) {Pulse};
\node [font=\small] at (1.5,16.75) {Generator};
\node [font=\small] at (4,17) {Stepper};
\node [font=\small] at (4,16.75) {motor};
\node [font=\small] at (5.75,17) {Gear Box};
\node [font=\small] at (5.75,16.75) {$U$};
\node [font=\small] at (9,17.5) {Table};
\node [font=\small] at (8.75,16.75) {Nut};
\node [font=\small] at (10.5,16) {Lead screw};
\draw [dashed] (4.5,17) -- (6.5,17);
\draw [short] (2.25,17.25) -- (2.5,17.25);
\draw [short] (2.5,17.25) -- (2.5,17.5);
\draw [short] (2.5,17.5) -- (2.75,17.5);
\draw [short] (2.75,17.5) -- (2.75,17.25);
\draw [short] (2.75,17.25) -- (3,17.25);
\draw [short] (3,17.25) -- (3,17.5);
\draw [short] (3,17.5) -- (3.25,17.5);
\draw [short] (3.25,17.5) -- (3.25,17.25);
\draw [short] (3.25,17.25) -- (3.5,17.25);
\node [font=\small] at (2.75,17.75) {$f$};
\end{circuitikz}
}
\end{figure}

 \item An $8$-bit, unipolar Successive Approximation Register type $ADC$
 is used to convert $3.5V$
 to digital equivalent output. The reference voltage is $+5V$
 The output of the $ADC$
 at the end of $3^{rd}$ clock pulse after the start of conversion, is \hfill(2015-EE)
 \begin{enumerate}
     \begin{multicols}{2}
         \item 1010 0000
         \item 1000 0000
         \item 0000 0001
         \item 0000 0011
     \end{multicols}
 \end{enumerate}
 \item Consider the economic dispatch problem for a power plant having two generating units. The fuel costs in $Rs/MWh$ along with the generating limits for the two units are given below: \\
 \begin{center}
     $C_1\brak{P_1} = 0.01P_1^2 + 30P_1 + 10$ ; $100MW \leq P_1 \leq 150MW$ \\
     $C_2\brak{P_2} = 0.05P_2^2 + 10P_2 + 10$ ; $100MW \leq P_2 \leq 180MW$      
 \end{center} 
 
 The incremental costs (in $Rs/MWh$) of the power plant when it supplies $200MW$ is 

 \hfill(2015-EE)
 \item Determine the correctness or otherwise of the following Assertion $\sbrak{\text{a}}$ and the Reason $\sbrak{r}$ \\ \\
 Assertion: Fast decoupled load flow method gives approximate load flow solution because it uses several assumptions. \\ \\
 Reason: Accuracy depends on the power mismatch vector tolerance. \hfill(2015-EE)
 \begin{enumerate}
     \item Both $\sbrak{\text{a}}$ and $\sbrak{\text{r}}$ are true and $\sbrak{\text{r}}$ is the correct reason for $\sbrak{\text{a}}$.
     \item Both $\sbrak{\text{a}}$ and $\sbrak{\text{r}}$ are true but $\sbrak{\text{r}}$ is not correct reason for $\sbrak{\text{a}}$.
     \item Both $\sbrak{\text{a}}$ and $\sbrak{\text{r}}$ are false
     \item $\sbrak{\text{a}}$ is false and $\sbrak{\text{r}}$ is true
 \end{enumerate}
\end{enumerate}
\end{document}
