\let\negmedspace\undefined
\let\negthickspace\undefined
\documentclass[journal]{IEEEtran}
\usepackage[a5paper, margin=10mm, onecolumn]{geometry}
\usepackage{lmodern} % Ensure lmodern is loaded for pdflatex
\usepackage{tfrupee} % Include tfrupee package

\setlength{\headheight}{1cm} % Set the height of the header box
\setlength{\headsep}{0mm}     % Set the distance between the header box and the top of the text

\usepackage{gvv-book}
\usepackage{gvv}
\usepackage{cite}
\usepackage{amsmath,amssymb,amsfonts,amsthm}
\usepackage{algorithmic}
\usepackage{graphicx}
\usepackage{textcomp}
\usepackage{xcolor}
\usepackage{txfonts}
\usepackage{listings}
\usepackage{enumitem}
\usepackage{mathtools}
\usepackage{gensymb}
\usepackage{comment}
\usepackage[breaklinks=true]{hyperref}
\usepackage{tkz-euclide} 
\usepackage{listings}
% \usepackage{gvv}                                        
\def\inputGnumericTable{}                                 
\usepackage[latin1]{inputenc}                                
\usepackage{color}                                            
\usepackage{array}                                            
\usepackage{longtable}                                       
\usepackage{calc}                                             
\usepackage{multirow}                                         
\usepackage{hhline}                                           
\usepackage{ifthen}                                           
\usepackage{lscape}
\begin{document}

\bibliographystyle{IEEEtran}
\vspace{3cm}

\title{2017-XE-53-65}
\author{EE24BTECH11010 - BALAJI B}
% \maketitle
% \newpage
% \bigskip
{\let\newpage\relax\maketitle}

\renewcommand{\thefigure}{\theenumi}
\renewcommand{\thetable}{\theenumi}
\setlength{\intextsep}{10pt} % Space between text and floats


\numberwithin{equation}{enumi}
\numberwithin{figure}{enumi}
\renewcommand{\thetable}{\theenumi}
\begin{enumerate}
    \item Copper is an $FCC$ metal with lattice parameter of 3.62 \AA . Hall effect measurement shows electron mobility to be $3.2 \times 10^{-3} {m}^2 {V}^{-1}{s}^{-1}$. Electrical resistivity of copper is $1.7 \times 10^{-8}  \Omega {m}$. The average number of free electrons per atom in copper is \rule{2.5cm}{0.5pt}(Charge of an electron: $1.6 \times 10^{-19}$ C) \hfill(2017-XE)
    \item In an ionic solid the cation and the anion have ionic radii as 0.8 \AA \ and 1.6 \AA \ respectively. The maximum coordination number of the cation in the structure will be \hfill(2017-XE)
 \begin{enumerate}
     \begin{multicols}{4}
         \item 3
         \item 4
         \item 6
         \item 8
     \end{multicols}
 \end{enumerate}
\item Which of the following statement(s) is / are true regarding susceptibility of a material
\begin{enumerate}[label=\roman*.]
    \item  Magnetic susceptibility is positive for a diamagnetic material 
    \item  Magnetic susceptibility is negative for a diamagnetic material 
    \item  Magnetic susceptibility is negative for an ferromagnetic material 
    \item Magnetic susceptibility is positive for a paramagnetic material
\end{enumerate}
\hfill(2017-XE)
\begin{enumerate}
    \begin{multicols}{4}
        \item (ii) and (iv)\\
        \item (i) and (iii)\\
        \item (ii) and (iii)\\
        \item (i) and (iv)
    \end{multicols}
\end{enumerate}
\item In the truss shown, a mass $m = 10kg$ is hung from the node J. The magnetic of net force(in Newtons) transferred by the truss 
 EFGHIJ onto the truss JKLMNO at the node J is \rule{2.5cm}{0.6pt} \\\\ Assume acceleration due to gravity $g = 10m/s^2$\hfill(2017-XE)
\begin{figure}[H]
\centering
\resizebox{10cm}{!}{%
\begin{circuitikz}
\tikzstyle{every node}=[font=\normalsize]
\draw (6,14) to[R] (8,14);
\draw (4.5,14) to[L ] (6.5,14);
\draw (8,14) to[short] (9,14);
\draw (9,14) to[short] (9,13.5);
\draw (9,13.5) to[european resistor] (9,12);
\draw (9,12) to[short] (4.5,12);
\draw (4.5,14.5) to[short] (4.5,11.75);
\draw (4.5,14.5) to[short] (1.75,14.5);
\draw (1.75,14.5) to[short] (1.75,11.75);
\draw (1.75,11.75) to[short] (4.5,11.75);
\draw [short] (3,14) -- (3,13.5);
\draw [short] (2.5,13.5) -- (3.5,13.5);
\draw [short] (3,13.5) -- (2.5,13);
\draw [short] (3,13.5) -- (3.5,13);
\draw [short] (2.5,13) -- (3.5,13);
\draw [short] (3,13) -- (3,12.5);
\draw [short] (3.25,13.5) -- (3.5,13.75);
\node [font=\normalsize] at (5.5,14.75) {Filter};
\node [font=\normalsize] at (7,14.75) {Choke};
\node [font=\normalsize] at (10,12.75) {Battery};
\draw [short] (-2,14.75) -- (-2,11.75);
\draw [short] (-2.5,14.75) -- (-2.5,11.75);
\draw (-3.25,13.25) to[tmultiwire] (-6.75,13.25);
\draw (1.75,13.25) to[tmultiwire] (-1.25,13.25);
\draw  (-7.5,13.25) circle (0.75cm);
\draw [short] (-8,13.25) .. controls (-7.5,13.75) and (-7.5,12.5) .. (-7,13.25);
\draw [decorate, decoration={coil, aspect=0.4, segment length=6pt, amplitude=10pt}] (-1.5,12) -- (-1.5,14.5);
\draw [decorate, decoration={coil, aspect=0.4, segment length=6pt, amplitude=10pt}] (-3,12) -- (-3,14.5);
\end{circuitikz}
}%

\label{fig:my_label}
\end{figure}
\item A ball moves along a plannar frictionless slot as shown. Which one of the paths shown closely matches the path by the ball after it exits the slot at E \hfill(2017-XE)
\begin{enumerate}
    \begin{multicols}{4}
        \item path $m$
        \item path $n$
        \item path $p$
        \item path $q$
    \end{multicols}
\end{enumerate}
\begin{figure}[!ht]
\centering
\resizebox{6cm}{!}{%
\begin{circuitikz}
\tikzstyle{every node}=[font=\normalsize]
\draw [line width=0.5pt, short] (3,12.5) .. controls (2.5,14.5) and (3,14.75) .. (4.25,14.25);
\draw [line width=0.5pt, short] (2.75,12.5) .. controls (2.25,13.75) and (2.25,15.25) .. (4,14.75);
\draw [line width=0.5pt, short] (4,14.75) -- (6.5,14);
\draw [line width=0.5pt, short] (4.25,14.25) -- (6.5,13.5);
\draw [line width=0.5pt, short] (6.5,13.5) .. controls (8.25,13.5) and (8.75,13.75) .. (8.75,14.5);
\draw [line width=0.5pt, short] (6.5,14) .. controls (8,13.75) and (8.5,14) .. (8.5,14.75);
\draw [line width=0.5pt, short] (8.75,14.5) .. controls (9,15.25) and (8.75,15.5) .. (8,16.25);
\draw [line width=0.5pt, short] (8.5,14.75) .. controls (8.25,15.5) and (8.25,15.75) .. (7.5,16);
\draw [line width=0.5pt, short] (7.5,16) .. controls (7.75,16) and (8,16) .. (8,16.25);
\draw [line width=0.5pt, short] (7.5,16) .. controls (7.75,16.25) and (7.75,16.25) .. (8,16.25);
\draw [line width=0.5pt, ->, >=Stealth, dashed] (8,16.25) -- (8.5,18);
\draw [line width=0.5pt, ->, >=Stealth, dashed] (8,16.25) .. controls (7,17) and (6.5,17) .. (6.25,18.25);
\draw [line width=0.5pt, ->, >=Stealth, dashed] (7.75,16.25) -- (5.25,17);
\draw [line width=0.5pt, ->, >=Stealth, dashed] (7.75,16) .. controls (6.25,16.25) and (6.25,16.25) .. (5.25,15.75);
\node [font=\normalsize] at (8.25,16.25) {$E$};
\node [font=\normalsize] at (6.75,18) {$p$};
\node [font=\normalsize] at (8.75,18) {$q$};
\node [font=\normalsize] at (6,17.25) {$n$};
\node [font=\normalsize] at (6,15.75) {$m$};
\end{circuitikz}
}%

\label{fig:my_label}
\end{figure}

\item A rod $EF$ moving in a plane has velocity $V_E$ at $E$ and $V_F$ that are parallel to each other. Which of the following \textbf{CANNOT} be true? \hfill(2017-XE)
\begin{figure}[!ht]
\centering
\resizebox{6.5cm}{!}{%
\begin{circuitikz}
\tikzstyle{every node}=[font=\large]
\draw [line width=0.7pt, short] (3.25,13.25) -- (10.25,11.5);
\draw [line width=0.7pt, short] (3.25,13.25) -- (3.5,13.5);
\draw [line width=0.7pt, short] (3.5,13.5) -- (10.5,11.75);
\draw [line width=0.7pt, short] (10.5,11.75) -- (10.25,11.5);
\draw [line width=0.7pt, ->, >=Stealth] (3.5,13.5) -- (4.5,14.25);
\draw [line width=0.7pt, ->, >=Stealth] (10.5,11.75) -- (11.75,13);
\node [font=\large] at (3.25,12.75) {$E$};
\node [font=\large] at (10,11.25) {$F$};
\node [font=\large] at (4,14.25) {$V_E$};
\node [font=\large] at (12,12.5) {$V_F$};
\end{circuitikz}
}%

\label{fig:my_label}
\end{figure}
\begin{enumerate}
    \item Both $V_E$ and $V_F$ are perpendicular to $EF$.
    \item Magnitude of $V_E$ is equal to the magnitude of $V_F$ and the angular velocity of $EF$ is zero.
    \item The velocity $V_E$ is not perpendicular to $EF$ and the angular velocity of $EF$ is nonzero.
    \item Magnitude of $V_E$ is not equal to the magnitude of $V_F$ and the angular velocity of $EF$ is nonzero.
\end{enumerate}
\item The beam shown below carries two external moments. A counterclockwise moment of magnitude $2M$ acts at point $B$ and a clockwise moment of magnitude $M$ acts at the free end, $C$. The beam is fixed at $A$. The shear force at a section close to the fixed end is equal to \hfill(2017-XE)
\begin{figure}[!ht]
\centering
\resizebox{7cm}{!}{%
\begin{circuitikz}
\tikzstyle{every node}=[font=\large]
\draw [ line width=0.7pt](-3,15.75) to[american voltage source] (-3,13);
\draw [ line width=0.7pt](-3,15.75) to[short] (-1.5,15.75);
\draw [ line width=0.7pt](-3,13) to[short] (-1.5,13);
\draw [ line width=0.7pt](-1.5,16.5) to[short] (-1.5,12.25);
\draw [ line width=0.7pt](-1.5,16.5) to[short] (2.25,16.5);
\draw [ line width=0.7pt](2.25,16.5) to[short] (2.25,12.25);
\draw [ line width=0.7pt](-1.5,12.25) to[short] (2.25,12.25);
\draw [ line width=0.7pt](2.25,13.25) to[short] (5.25,13.25);
\draw [line width=0.7pt](5.25,13.25) to[C] (5.25,15.75);
\draw [line width=0.7pt](2.25,15.75) to[L ] (5.25,15.75);
\draw [ line width=0.7pt](5.25,15.75) to[short] (6.75,15.75);
\draw [ line width=0.7pt](5.25,13.25) to[short] (6.75,13.25);
\draw [ line width=0.7pt](6.75,15.75) to[R] (6.75,13.25);
\node [font=\large] at (-3.75,14.25) {$V_{in}$};
\node [font=\large] at (0.5,14.5) {Full-Bridge};
\node [font=\large] at (0.25,13.75) {VSI};
\node [font=\large] at (2.5,14.5) {$V_R$};
\node [font=\large] at (2.5,16) {$+$};
\node [font=\LARGE] at (2.5,13) {$-$};
\node [font=\large] at (4.5,14.5) {$C$};
\node [font=\large] at (6.25,14.5) {$R$};
\node [font=\large] at (7.25,14.5) {$V_o$};
\node [font=\large] at (7,15.5) {$+$};
\node [font=\LARGE] at (7,13.25) {$-$};
\node [font=\large] at (3.75,16.5) {$L$};
\end{circuitikz}
}%

\label{fig:my_label}
\end{figure}

\begin{enumerate}
    \begin{multicols}{4}
        \item $\frac{2M}{L}$
        \item $\frac{M}{L}$
        \item 0
        \item $-\frac{M}{L}$
    \end{multicols}
\end{enumerate}
\item Two pendulums are shown below. \textbf{\textit{Pendulum-A}} carries a bob of mass $m,$ hung using a hinged massless rigid rod of length $L$ whereas \textbf{\textit{Pendulum-B}} carries a bob of mass $4m$ and length $L/4$. The ratio of the natural frequencies of \textbf{\textit{Pendulum-A}} and \textbf{\textit{Pendulum-B}} is given by \hfill(2017-XE)
\begin{figure}[H]
\centering
\resizebox{2.5cm}{!}{%
\begin{circuitikz}
\tikzstyle{every node}=[font=\Large]
\node [font=\Large, rotate around={180:(0,0)}] at (1.75,14.25) {T};
\node [font=\Large, rotate around={180:(0,0)}] at (2.15,14.25) {R};
\node [font=\Large] at (2.5,14.25) {I};
\node [font=\Large, rotate around={180:(0,0)}] at (2.75,14.25) {A};
\node [font=\Large, rotate around={-180:(0,0)}] at (3.25,14.25) {N};
\node [scale=-1, yscale=-1, font=\Large, rotate around={180:(0,0)}] at (3.75,14.25) {G};
\node [font=\Large, rotate around={180:(0,0)}] at (4.15,14.25) {L};
\node [font=\Large] at (4.5,14.25) {E};
\end{circuitikz}
}%

\label{fig:my_label}
\end{figure}

\begin{enumerate}
    \begin{multicols}{4}
        \item 1 : 2
        \item 1 : 1
        \item $\sqrt{2}$ : 1
        \item 2 : 1 
    \end{multicols}
\end{enumerate}
\item A closed thin-walled cylindrical steel pressure vessel of wall thickness $t = 1 \ \text{mm}$ is subjected to internal pressure. The maximum value of pressure $p$ (in kPa) that the wall can withstand based on the maximum shear stress failure theory is given by (Yield strength of steel is $200  MPa$ and mean radius of the cylinder $r = 1 m$).

\hfill(2017-XE)
\begin{enumerate}
    \begin{multicols}{4}
        \item 100
        \item 200
        \item 300
        \item 400
    \end{multicols}
\end{enumerate}
\item The state of stress at a point in a body is represented using components of stresses along $X$ and $Y$ directions as shown. Which one of the following represents the state of the stress along $X^\prime$ and $Y^\prime$ axes?($X^\prime$- axis at $45^\degree$ clockwise with respect to $X$- axis)

\hfill(2017-XE)
\begin{figure}[H]
\centering
\resizebox{6cm}{!}{%
\begin{circuitikz}
\tikzstyle{every node}=[font=\normalsize]

\draw [line width=0.5pt, short] (3.75,13.5) -- (3.75,11.25);
\draw [line width=0.5pt, short] (3.75,13.5) -- (6.5,13.5);
\draw [line width=0.5pt, short] (6.5,13.5) -- (6.5,11.25);
\draw [line width=0.5pt, short] (3.75,11.25) -- (6.5,11.25);
\draw [line width=0.5pt, ->, >=Stealth] (2.25,12.5) -- (3.75,12.5);
\draw [line width=0.5pt, ->, >=Stealth] (8,12.5) -- (6.5,12.5);
\draw [line width=0.5pt, ->, >=Stealth, dashed] (5,12.5) -- (5,13.25);
\draw [line width=0.5pt, ->, >=Stealth, dashed] (5,12.5) -- (5.75,12.5);
\node [font=\normalsize] at (7,12.75) {$\sigma$};
\node [font=\normalsize] at (5.5,12.75) {$X$};
\node [font=\normalsize] at (4.75,13) {$Y$};
\end{circuitikz}
}%

\label{fig:my_label}
\end{figure}

\begin{enumerate}
    \item \input{figs/fig9}
    \item \input{figs/fig10}
    \item \input{figs/fig11}
    \item \begin{figure}[!ht]
\centering
\resizebox{2cm}{!}{%
\begin{circuitikz}
\tikzstyle{every node}=[font=\normalsize]
\draw [line width=0.5pt, short] (5,14) -- (3.5,12.5);
\draw [line width=0.5pt, short] (5,14) -- (6.5,12.5);
\draw [line width=0.5pt, short] (3.5,12.5) -- (5,11);
\draw [line width=0.5pt, short] (5,11) -- (6.5,12.5);
\draw [line width=0.5pt, short] (6.5,12.75) -- (5.25,14);
\draw [line width=0.5pt, short] (5.25,14) -- (5.5,14);
\draw [line width=0.5pt, short] (6.5,12.25) -- (5.25,11);
\draw [line width=0.5pt, short] (5.25,11) -- (5.75,11);
\draw [line width=0.5pt, short] (3.5,12.75) -- (4.75,14);
\draw [line width=0.5pt, short] (4.75,14) -- (4.5,14);
\draw [line width=0.5pt, short] (3.5,12.25) -- (4.75,11);
\draw [line width=0.5pt, short] (4.75,11) -- (4.5,11);
\draw [line width=0.5pt, ->, >=Stealth] (4,13.25) -- (3,14.25);
\draw [line width=0.5pt, ->, >=Stealth] (6,13.25) -- (6.75,14);
\draw [line width=0.5pt, ->, >=Stealth] (6,11.75) -- (6.75,11);
\draw [line width=0.5pt, ->, >=Stealth] (4,11.75) -- (3.25,11);
\draw [line width=0.5pt, ->, >=Stealth, dashed] (4.5,12.5) -- (5.25,13.25);
\draw [line width=0.5pt, ->, >=Stealth, dashed] (4.5,12.5) -- (5.25,11.75);
\node [font=\normalsize] at (5,13.25) {$Y$};
\node [font=\normalsize] at (5.5,12) {$X$};
\node [font=\normalsize] at (7.25,12.5) {$\tau = \sigma/2 $};
\node [font=\normalsize] at (7,13.5) {$\sigma/2$};
\node [font=\normalsize] at (6.75,11.75) {$\sigma/2$};
\node [font=\normalsize] at (5,10.75) {$\tau = \sigma/2$};
\node [font=\normalsize] at (3,12.5) {$\tau = \sigma/2 $};
\end{circuitikz}
}%

\label{fig:my_label}
\end{figure}

\end{enumerate}
\item An aluminum specimen with an initial gauge diameter $d_0 = 10mm$ and a gauge length $l_0 = 10mm$ is subjected to tension test. A tensile force $P = 50kN$ is applied at the ends of the specimen as shown resulting in an elongation of $1mm$ in the gauge length. The Poisson's ratio ($\gamma$) of the specimen is \rule{2cm}{0.4pt}\\ \\
Shear modulus of the material $G = 25GPa$. Consider engineering stress-strain conditions. \hfill(2017-XE)
\begin{figure}[H]
\centering
\resizebox{4cm}{!}{%
\begin{circuitikz}
\tikzstyle{every node}=[font=\large]
\draw [short] (1.25,23.75) -- (1.25,19.75);
\draw [short] (1.25,23.75) -- (2.75,23.75);
\draw [short] (2.75,23.75) -- (2.75,19.75);
\draw [short] (1.25,19.75) -- (2.75,19.75);
\draw [short] (0.75,24) -- (0.75,26.25);
\draw [short] (0.75,26.25) -- (3.25,26.25);
\draw [short] (3.25,26.25) -- (3.25,24);
\draw [short] (0.75,24) -- (3.25,24);
\draw [short] (1.25,26.25) -- (0.75,25.75);
\draw [short] (1.75,26.25) -- (0.75,25.25);
\draw [short] (2.25,26.25) -- (0.75,24.75);
\draw [short] (3,26.25) -- (0.75,24);
\draw [short] (3.25,26) -- (1.25,24);
\draw [short] (3.25,25.5) -- (1.75,24);
\draw [short] (3.25,25) -- (2.25,24);
\draw [short] (3.25,24.5) -- (2.75,24);
\draw [short] (2.75,26.25) -- (2.5,26.25);
\draw [short] (1.75,24) -- (1.75,23.75);
\draw [short] (2.5,24) -- (2.5,23.75);
\draw [short] (-1.25,17) -- (5.75,17);
\draw [short] (-0.75,17) -- (-1,16.25);
\draw [short] (0,17) -- (-0.25,16.25);
\draw [short] (0.75,17) -- (0.5,16.25);
\draw [short] (1.5,17) -- (1.25,16.25);
\draw [short] (2.25,17) -- (2,16.25);
\draw [short] (3,17) -- (2.75,16.25);
\draw [short] (3.75,17) -- (3.5,16.25);
\draw [short] (4.5,17) -- (4.25,16.25);
\draw [short] (5.25,17) -- (5,16.25);
\draw [dashed] (2,28) -- (2.25,14.5);
\draw  (4.75,23) circle (0.75cm);
\draw  (4.75,20.75) circle (0.75cm);
\draw [->, >=Stealth] (4.5,20.5) -- (5,21.25);
\draw [->, >=Stealth] (4.5,22.75) -- (5,23.5);
\draw [->, >=Stealth] (6.5,22) -- (5.5,22.75);
\draw [->, >=Stealth] (6.5,21.25) -- (5.5,20.75);
\draw [->, >=Stealth] (-0.75,24) -- (0.5,24.75);
\draw [->, >=Stealth] (0,21.75) -- (1,21.75);
\node [font=\Large] at (-1.25,24) {drill};
\node [font=\Large] at (-1.25,23.5) {spindle};
\node [font=\Large] at (-0.5,17.75) {drill};
\node [font=\Large] at (-0.5,17.25) {table};
\node [font=\Large] at (6.5,21.5) {sensor};
\node [font=\Large] at (6.25,22.75) {$P$};
\node [font=\Large] at (6.25,20.75) {$Q$};
\node [font=\Large] at (-0.75,21.75) {mandrel};
\draw [->, >=Stealth] (4,23) -- (2.75,23);
\draw [->, >=Stealth] (4,20.75) -- (2.75,20.75);
\draw [short] (1.75,25.25) -- (1.25,24);
\draw [short] (1.75,25.25) -- (2.25,25.25);
\draw [short] (2.25,25.25) -- (2.5,24);
\draw [->, >=Stealth] (2,18.75) .. controls (2.75,18.75) and (2.75,18.25) .. (2,18.25) ;
\end{circuitikz}
}
\label{fig:my_label}
\end{figure}

\item A rectangular sheet $ABCD$ of dimensions $a$ and $b$ along $X$ and $Y$ directions, respectively, is stretched to a rectangle $AB'C'D'$, as shown. The maximum principal strain ($\varepsilon_1$) and minimum principal strain ($\varepsilon_2$) due to the stretch are given by

\hfill(2017-XE)
\begin{figure}[H]
\centering
\resizebox{10cm}{!}{%
\begin{circuitikz}
\tikzstyle{every node}=[font=\small]

\draw [short] (1,17.25) -- (1,16.5);
\draw [short] (1,17.25) -- (2.25,17.25);
\draw [short] (2.25,17.25) -- (2.25,16.5);
\draw [short] (1,16.5) -- (2.25,16.5);
\draw [short] (3.5,17.25) -- (3.5,16.5);
\draw [short] (3.5,17.25) -- (4.5,17.25);
\draw [short] (4.5,17.25) -- (4.5,16.5);
\draw [short] (3.5,16.5) -- (4.5,16.5);
\draw [short] (5.25,17.25) -- (5.25,16.5);
\draw [short] (5.25,17.25) -- (6.25,17.25);
\draw [short] (6.25,17.25) -- (6.25,16.5);
\draw [short] (5.25,16.5) -- (6.25,16.5);
\draw [short] (4.5,17) -- (5.25,17);
\draw [short] (4.5,16.75) -- (5.25,16.75);
\draw [short] (6.25,17) -- (8.25,17);
\draw [short] (6.25,16.75) -- (8.25,16.75);
\draw [short] (8.25,17.25) -- (8.25,16.5);
\draw [short] (8.25,17.25) -- (9.5,17.25);
\draw [short] (9.5,17.25) -- (9.5,16.5);
\draw [short] (8.25,16.5) -- (9.5,16.5);
\draw [short] (7.25,17) -- (7.25,16.75);
\draw [short] (7.5,17) -- (7.25,16.75);
\draw [short] (7.75,17) -- (7.5,16.75);
\draw [short] (8,17) -- (7.75,16.75);
\draw [short] (8.25,17) -- (8,16.75);
\draw [short] (7,17.25) -- (10.75,17.25);
\draw [short] (7,17.25) -- (7,17.75);
\draw [short] (7,17.75) -- (10.75,17.75);
\draw [short] (10.75,17.75) -- (10.75,17.25);
\draw [short] (9.5,17) -- (10.75,17);
\draw [short] (9.5,16.75) -- (10.75,16.75);
\draw [short] (10.75,17) -- (10.75,16.75);
\draw [short] (9.75,17) -- (9.5,16.75);
\draw [short] (10,17) -- (9.75,16.75);
\draw [short] (10.25,17) -- (10,16.75);
\draw [short] (10.5,17) -- (10.25,16.75);
\draw [short] (10.75,17) -- (10.5,16.75);
\draw [->, >=Stealth] (2.25,17) -- (3.5,17);
\draw [->, >=Stealth] (10.25,16.25) -- (9.75,16.75);
\node [font=\small] at (1.5,17) {Pulse};
\node [font=\small] at (1.5,16.75) {Generator};
\node [font=\small] at (4,17) {Stepper};
\node [font=\small] at (4,16.75) {motor};
\node [font=\small] at (5.75,17) {Gear Box};
\node [font=\small] at (5.75,16.75) {$U$};
\node [font=\small] at (9,17.5) {Table};
\node [font=\small] at (8.75,16.75) {Nut};
\node [font=\small] at (10.5,16) {Lead screw};
\draw [dashed] (4.5,17) -- (6.5,17);
\draw [short] (2.25,17.25) -- (2.5,17.25);
\draw [short] (2.5,17.25) -- (2.5,17.5);
\draw [short] (2.5,17.5) -- (2.75,17.5);
\draw [short] (2.75,17.5) -- (2.75,17.25);
\draw [short] (2.75,17.25) -- (3,17.25);
\draw [short] (3,17.25) -- (3,17.5);
\draw [short] (3,17.5) -- (3.25,17.5);
\draw [short] (3.25,17.5) -- (3.25,17.25);
\draw [short] (3.25,17.25) -- (3.5,17.25);
\node [font=\small] at (2.75,17.75) {$f$};
\end{circuitikz}
}
\end{figure}

\begin{enumerate}
    \begin{multicols}{2}
        \item $\varepsilon_1 = 0.001$ and $\varepsilon_2 = 0.001$
        \item $\varepsilon_1 = -0.001$ and $\varepsilon_2 = 0.001$
        \item $\varepsilon_1 = 0.001$ and $\varepsilon_2 = -0.001$
        \item  $\varepsilon_1 = -0.001$ and $\varepsilon_2 = -0.001$
    \end{multicols}
\end{enumerate}
\item A solid bar of uniform square cross-section of side $b$ and length $L$ is rigidly fixed to the supports at the two ends. When the temperature in the rod is increased uniformly by $T$, the bar undergoes elastic buckling. Assume Young's modulus $E$ and coefficient of thermal expansion $\alpha$ to be independent of temperature. The coefficient of thermal expansion $\alpha$ is given by \hfill(2017-XE)
\begin{enumerate}
    \begin{multicols}{4}
        \item $\frac{3\pi^2b^2}{T_cL^2}$
         \item $\frac{\pi^2b^2}{T_cL^2}$
          \item $\frac{\pi^2b^2}{2T_cL^2}$
           \item $\frac{\pi^2b^2}{3T_cL^2}$
    \end{multicols}
\end{enumerate}
\end{enumerate}
\end{document}