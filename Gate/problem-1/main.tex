\let\negmedspace\undefined
\let\negthickspace\undefined
\documentclass[journal]{IEEEtran}
\usepackage[a5paper, margin=10mm, onecolumn]{geometry}
%\usepackage{lmodern} % Ensure lmodern is loaded for pdflatex
\usepackage{tfrupee} % Include tfrupee package

\setlength{\headheight}{1cm} % Set the height of the header box
\setlength{\headsep}{0mm}     % Set the distance between the header box and the top of the text

\usepackage{gvv-book}
\usepackage{gvv}
\usepackage{cite}
\usepackage{amsmath,amssymb,amsfonts,amsthm}
\usepackage{algorithmic}
\usepackage{graphicx}
\usepackage{textcomp}
\usepackage{xcolor}
\usepackage{txfonts}
\usepackage{listings}
\usepackage{enumitem}
\usepackage{mathtools}
\usepackage{gensymb}
\usepackage{comment}
\usepackage[breaklinks=true]{hyperref}
\usepackage{tkz-euclide} 
\usepackage{listings}
% \usepackage{gvv}                                        
\def\inputGnumericTable{}                                 
\usepackage[latin1]{inputenc}                                
\usepackage{color}                                            
\usepackage{array}                                            
\usepackage{longtable}                                       
\usepackage{calc}                                             
\usepackage{multirow}                                         
\usepackage{hhline}                                           
\usepackage{ifthen}                                           
\usepackage{lscape}
\begin{document}

\bibliographystyle{IEEEtran}
\vspace{3cm}

\title{2007-PH-1-17}
\author{EE24BTECH11010 - BALAJI B}
% \maketitle
% \newpage
% \bigskip
{\let\newpage\relax\maketitle}

\renewcommand{\thefigure}{\theenumi}
\renewcommand{\thetable}{\theenumi}
\setlength{\intextsep}{10pt} % Space between text and floats
\numberwithin{equation}{enumi}
\numberwithin{figure}{enumi}
\renewcommand{\thetable}{\theenumi}
\begin{enumerate}
    \item The eigenvalues of a matrix are $i, -2i$ and $3i$. The matrix is 
    \begin{enumerate}
        \begin{multicols}{2}
            \item unitary
            \item anti-unitary
            \item Hermitian 
            \item anti-Hermitian
        \end{multicols}
    \end{enumerate}
    \item A space station moving in a circular orbit around the Earth goes into a new bound orbit by firing its engine radially outwards. The orbit is 
    \begin{enumerate}
        \begin{multicols}{2}
            \item A larger circle
            \item a smaller circle
            \item an ellipse
            \item a parabola
        \end{multicols}
    \end{enumerate}
    \item A power amplifier gives $150W$ output for an input of $1.5W$. The gain, in $dB$, is 
    \begin{enumerate}
        \begin{multicols}{2}
            \item 10
            \item 20
            \item 54
            \item 100
        \end{multicols}
    \end{enumerate}
    \item Four point charges are placed in a plane at the following positions: $+Q$ at $\brak{1,0}$, $-Q$ at $\brak{-1,0}$, $+Q$ at $\brak{0,1}$ and $-Q$ at $\brak{0,-1}$. At large distances the electrostatic potential due to this charge distribution will be dominated by the 
    \begin{enumerate}
        \begin{multicols}{2}
            \item monopole moment 
            \item dipole moment 
            \item quadrupole moment 
            \item octopole moment 
        \end{multicols}
    \end{enumerate}
    \item A charged capacitor $\brak{C}$ is connected in series with an inductor $\brak{L}$. When the displacement current reduces to zero, the energy of the $LC$ circuit is
    \begin{enumerate}
            \item stored entirely in its magnetic field.
            \item stored entirely in its electric field
            \item distributed equally among its electric and magnetic fields
            \item radiated out of the circuit.
    \end{enumerate}
    \item Match the following 
    \begin{table}[!h]
        \centering
        \begin{tabular}[12pt]{ |c|c|c|c|c|c|}
    \hline
     & 0 & 0 & 0 & 2 & 12 \\
    \hline
    $s$ & $\frac{5}{3}$ & 0 & 1 & $-\frac{1}{3}$ & 2\\
    \hline 
    $y$ & $\frac{2}{3}$ & 1 & 0 & $\frac{1}{3}$ & 2\\
    \hline
     & $x$ & $y$ & $s$ & $t$ &RHS\\
    \hline
    \end{tabular}

    \end{table} 
    
    \begin{table}[H]
        \centering
        \begin{tabular}[12pt]{ |c|c|c|c|c|c|}
    \hline
     & -4 & -6 & 0 & 0 & 0 \\
    \hline
    $s$ & 3 & 2 & 1 & 0 & 6\\
    \hline 
    $t$ & 2 & 3 & 0 & 1 & 6\\
    \hline
     & $x$ & $y$ & $s$ & $t$ &RHS\\
    \hline
    \end{tabular}

    \end{table} 
    \item The wave function of a particle, moving in a one-dimensional time-independent potential  $V(x)$, is given by $\Psi(x) = e^{-iax + b}$, where $a$ and $b$ are constants. This means that the potential $V(x)$ is of the form 
    \begin{enumerate}
        \begin{multicols}{2}
            \item $V(x) \propto x$
            \item $V(x) \propto x^2$
            \item $V(x) = 0 $
            \item $V(x) \propto e^{-ax}$
        \end{multicols}
    \end{enumerate}
    \item The $D_1$ and $D_2$ lines of $Na\brak{3^2P_{\frac{1}{2}} \to 3^2S_{\frac{1}{2}}, 3^2P_{\frac{3}{2}} \to 3^2S_{\frac{1}{2}}}$ will split on the application of a weak magnetic field magnetic field into 
    \begin{enumerate}
        \begin{multicols}{2}
            \item 4 and 6 lines respectively
            \item 3 lines each 
            \item 6 and 4 lines respectively
            \item 6 lines each
        \end{multicols}
    \end{enumerate}
    \item In a $He-Ne$ laser transition takes place in 
    \begin{enumerate}
        \begin{multicols}{2}
            \item $He$ only
            \item $Ne$ only
            \item $Ne$ first, then in $He$
            \item $He$ first, then in $Ne$
        \end{multicols}
    \end{enumerate}
    \item The partition function of a single gas molecule is $Z_{\alpha}$. The partition function of $N$ such non-interacting gas molecules is then given by
    \begin{enumerate}
        \begin{multicols}{2}
            \item $\frac{\brak{Z_a}^N}{N!}$
            \item $\brak{Z_a}^N$
            \item $N\brak{Z_{\alpha}}$
            \item $\frac{\brak{Z_{\alpha}}^N}{N}$
        \end{multicols}
    \end{enumerate}
    \item A solid superconductor is placed in an external magnetic field and then cooled below its critical temperature. The superconductor 
    \begin{enumerate}
        \item retains its magnetic flux because the surface current supports it.
        \item expels out its magnetic flux because it behaves like a paramagnetic material 
        \item expels out its magnetic flux because it behaves like an anti-ferromagnetic material 
        \item expels out its magnetic flux because the surface current induces a field in the opposite to the applied magnetic field 
    \end{enumerate}
    \item A particle with energy $E$ is a time-independent double well potential as shown in the figure. 
    \begin{figure}[!h]
\centering
\resizebox{2cm}{!}{%
\begin{circuitikz}
\tikzstyle{every node}=[font=\huge]

\draw [->, >=Stealth, line width = 1pt] (0,-4) -- (0,4);
\draw [->, >=Stealth, line width = 1pt] (-4,0) -- (4,0);
\node [font=\Huge] at (.5,3.8) {V};
\draw [dashed] (-3,-3) -- (3,-3);
\node [font=\Huge] at (-3.5,-3) {E};
\draw (-.7,-.62) to[short] (.7,-.62);
\node[font = \Huge] at (4.5,0) {x};
\draw[thick, black, samples=50, domain=0.7:3.8] 
        plot (\x, {2*(\x - 2)^2 - 4});
\draw[thick, black, samples=50, domain=-3.8:-0.7] 
        plot (\x, {2*(\x + 2)^2 - 4});
\end{circuitikz}
}%
\end{figure}
 
    
    Which of the following statements about the particle is \textbf{NOT} correct?
    \begin{enumerate}
            \item The particle will always be in a bound state  
            \item The probability of finding the particle in one well will be time-dependent
            \item The particle will be confined to any one of the wells 
            \item The particle can tunnel from one well to the other, and back.
    \end{enumerate}
    \item It is necessary to apply quantum statistics to a system of particles if 
    \begin{enumerate}
        \item there is substantial overlap between the wavefunctions of the particles
        \item the mean free path of the particles is comparable to the inner-particle seperation.
        \item the particle have identical mass and charge 
        \item the particle are interacting.
\end{enumerate}
\item When liquid oxygen is poured down close to a strong bar magnet, the oxygen stream is 
\begin{enumerate}
    \item repelled towards the field because it is diamagnetic.
    \item attracted towards the higher field because it is diamagnetic.
    \item repelled towards the lower field because it is paramagnetic.
    \item attracted towards the higher field because it is paramagnetic.
\end{enumerate}
\item Fission fragments are generally radioactive as 
\begin{enumerate}
    \item they have excess of neutrons. 
    \item they have excess of protons.
    \item they are products of radioactive nuclides.
    \item their total kinetic energy is of the order of $200MeV$.
\end{enumerate}
\item In a typical $npn$ transistor the doping concentrations in emitter, base and collector regions are $C_E, C_B$ and $C_E$ respectively. These satisfy the relation 
\begin{enumerate}
    \begin{multicols}{2}
        \item $C_E > C_C > C_B$ 
        \item $C_E > C_B > C_C$
        \item $C_C > C_B > C_E$
        \item $C_E = C_C > C_B$
    \end{multicols}
\end{enumerate}
\item The allowed states for $He\brak{2p^2}$ configuration are 
\begin{enumerate}
        \item $^1S_0$, $^3 S_1$, $^1P_1$, $^3P_{0,1,2}$, $^1D_2$ and $ ^3D_{1,2,3}$
        \item  $^1S_0$, $^3P_{0,1,2}$ and $^1D_2$
        \item $^1P_1$ and $^3P_{0,1,2}$
        \item $^1S_0$ and $^1P_1$
\end{enumerate}
\end{enumerate}
\end{document}
