\let\negmedspace\undefined
\let\negthickspace\undefined
\documentclass[journal]{IEEEtran}
\usepackage[a5paper, margin=10mm, onecolumn]{geometry}
\usepackage{lmodern} % Ensure lmodern is loaded for pdflatex
\usepackage{tfrupee} % Include tfrupee package

\setlength{\headheight}{1cm} % Set the height of the header box
\setlength{\headsep}{0mm}     % Set the distance between the header box and the top of the text

\usepackage{gvv-book}
\usepackage{gvv}
\usepackage{cite}
\usepackage{amsmath,amssymb,amsfonts,amsthm}
\usepackage{algorithmic}
\usepackage{graphicx}
\usepackage{textcomp}
\usepackage{xcolor}
\usepackage{txfonts}
\usepackage{listings}
\usepackage{enumitem}
\usepackage{mathtools}
\usepackage{gensymb}
\usepackage{comment}
\usepackage[breaklinks=true]{hyperref}
\usepackage{tkz-euclide} 
\usepackage{listings}
% \usepackage{gvv}                                        
\def\inputGnumericTable{}                                 
\usepackage[latin1]{inputenc}                                
\usepackage{color}                                            
\usepackage{array}                                            
\usepackage{longtable}                                       
\usepackage{calc}                                             
\usepackage{multirow}                                         
\usepackage{hhline}                                           
\usepackage{ifthen}                                           
\usepackage{lscape}
\begin{document}

\bibliographystyle{IEEEtran}
\vspace{3cm}

\title{2021-CE-1-13}
\author{EE24BTECH11010 - BALAJI B}
% \maketitle
% \newpage
% \bigskip
{\let\newpage\relax\maketitle}

\renewcommand{\thefigure}{\theenumi}
\renewcommand{\thetable}{\theenumi}
\setlength{\intextsep}{10pt} % Space between text and floats


\numberwithin{equation}{enumi}
\numberwithin{figure}{enumi}
\renewcommand{\thetable}{\theenumi}
\begin{enumerate}
    \item Getting to the top is \rule{2cm}{0.4pt} than staying on the top \hfill [2021 CE]
    \begin{enumerate}
        \begin{multicols}{2}
            \item more easy 
            \item much easy
            \item easiest
            \item easier
        \end{multicols}
    \end{enumerate}
    \item The mirror image of the below text above the $x-$ axis is \hfill [2021 CE]
    \begin{figure}[H]
\centering
\resizebox{10cm}{!}{%
\begin{circuitikz}
\tikzstyle{every node}=[font=\normalsize]
\draw (6,14) to[R] (8,14);
\draw (4.5,14) to[L ] (6.5,14);
\draw (8,14) to[short] (9,14);
\draw (9,14) to[short] (9,13.5);
\draw (9,13.5) to[european resistor] (9,12);
\draw (9,12) to[short] (4.5,12);
\draw (4.5,14.5) to[short] (4.5,11.75);
\draw (4.5,14.5) to[short] (1.75,14.5);
\draw (1.75,14.5) to[short] (1.75,11.75);
\draw (1.75,11.75) to[short] (4.5,11.75);
\draw [short] (3,14) -- (3,13.5);
\draw [short] (2.5,13.5) -- (3.5,13.5);
\draw [short] (3,13.5) -- (2.5,13);
\draw [short] (3,13.5) -- (3.5,13);
\draw [short] (2.5,13) -- (3.5,13);
\draw [short] (3,13) -- (3,12.5);
\draw [short] (3.25,13.5) -- (3.5,13.75);
\node [font=\normalsize] at (5.5,14.75) {Filter};
\node [font=\normalsize] at (7,14.75) {Choke};
\node [font=\normalsize] at (10,12.75) {Battery};
\draw [short] (-2,14.75) -- (-2,11.75);
\draw [short] (-2.5,14.75) -- (-2.5,11.75);
\draw (-3.25,13.25) to[tmultiwire] (-6.75,13.25);
\draw (1.75,13.25) to[tmultiwire] (-1.25,13.25);
\draw  (-7.5,13.25) circle (0.75cm);
\draw [short] (-8,13.25) .. controls (-7.5,13.75) and (-7.5,12.5) .. (-7,13.25);
\draw [decorate, decoration={coil, aspect=0.4, segment length=6pt, amplitude=10pt}] (-1.5,12) -- (-1.5,14.5);
\draw [decorate, decoration={coil, aspect=0.4, segment length=6pt, amplitude=10pt}] (-3,12) -- (-3,14.5);
\end{circuitikz}
}%

\label{fig:my_label}
\end{figure}
    \begin{enumerate}
        \item \begin{figure}[!ht]
\centering
\resizebox{6.5cm}{!}{%
\begin{circuitikz}
\tikzstyle{every node}=[font=\large]
\draw [line width=0.7pt, short] (3.25,13.25) -- (10.25,11.5);
\draw [line width=0.7pt, short] (3.25,13.25) -- (3.5,13.5);
\draw [line width=0.7pt, short] (3.5,13.5) -- (10.5,11.75);
\draw [line width=0.7pt, short] (10.5,11.75) -- (10.25,11.5);
\draw [line width=0.7pt, ->, >=Stealth] (3.5,13.5) -- (4.5,14.25);
\draw [line width=0.7pt, ->, >=Stealth] (10.5,11.75) -- (11.75,13);
\node [font=\large] at (3.25,12.75) {$E$};
\node [font=\large] at (10,11.25) {$F$};
\node [font=\large] at (4,14.25) {$V_E$};
\node [font=\large] at (12,12.5) {$V_F$};
\end{circuitikz}
}%

\label{fig:my_label}
\end{figure}
        \item \begin{figure}[!ht]
\centering
\resizebox{7cm}{!}{%
\begin{circuitikz}
\tikzstyle{every node}=[font=\large]
\draw [ line width=0.7pt](-3,15.75) to[american voltage source] (-3,13);
\draw [ line width=0.7pt](-3,15.75) to[short] (-1.5,15.75);
\draw [ line width=0.7pt](-3,13) to[short] (-1.5,13);
\draw [ line width=0.7pt](-1.5,16.5) to[short] (-1.5,12.25);
\draw [ line width=0.7pt](-1.5,16.5) to[short] (2.25,16.5);
\draw [ line width=0.7pt](2.25,16.5) to[short] (2.25,12.25);
\draw [ line width=0.7pt](-1.5,12.25) to[short] (2.25,12.25);
\draw [ line width=0.7pt](2.25,13.25) to[short] (5.25,13.25);
\draw [line width=0.7pt](5.25,13.25) to[C] (5.25,15.75);
\draw [line width=0.7pt](2.25,15.75) to[L ] (5.25,15.75);
\draw [ line width=0.7pt](5.25,15.75) to[short] (6.75,15.75);
\draw [ line width=0.7pt](5.25,13.25) to[short] (6.75,13.25);
\draw [ line width=0.7pt](6.75,15.75) to[R] (6.75,13.25);
\node [font=\large] at (-3.75,14.25) {$V_{in}$};
\node [font=\large] at (0.5,14.5) {Full-Bridge};
\node [font=\large] at (0.25,13.75) {VSI};
\node [font=\large] at (2.5,14.5) {$V_R$};
\node [font=\large] at (2.5,16) {$+$};
\node [font=\LARGE] at (2.5,13) {$-$};
\node [font=\large] at (4.5,14.5) {$C$};
\node [font=\large] at (6.25,14.5) {$R$};
\node [font=\large] at (7.25,14.5) {$V_o$};
\node [font=\large] at (7,15.5) {$+$};
\node [font=\LARGE] at (7,13.25) {$-$};
\node [font=\large] at (3.75,16.5) {$L$};
\end{circuitikz}
}%

\label{fig:my_label}
\end{figure}

        \item \begin{figure}[H]
\centering
\resizebox{2.5cm}{!}{%
\begin{circuitikz}
\tikzstyle{every node}=[font=\Large]
\node [font=\Large, rotate around={180:(0,0)}] at (1.75,14.25) {T};
\node [font=\Large, rotate around={180:(0,0)}] at (2.15,14.25) {R};
\node [font=\Large] at (2.5,14.25) {I};
\node [font=\Large, rotate around={180:(0,0)}] at (2.75,14.25) {A};
\node [font=\Large, rotate around={-180:(0,0)}] at (3.25,14.25) {N};
\node [scale=-1, yscale=-1, font=\Large, rotate around={180:(0,0)}] at (3.75,14.25) {G};
\node [font=\Large, rotate around={180:(0,0)}] at (4.15,14.25) {L};
\node [font=\Large] at (4.5,14.25) {E};
\end{circuitikz}
}%

\label{fig:my_label}
\end{figure}

        \item \begin{figure}[H]
\centering
\resizebox{4cm}{!}{%
\begin{circuitikz}
\tikzstyle{every node}=[font=\large]
\draw [short] (1.25,23.75) -- (1.25,19.75);
\draw [short] (1.25,23.75) -- (2.75,23.75);
\draw [short] (2.75,23.75) -- (2.75,19.75);
\draw [short] (1.25,19.75) -- (2.75,19.75);
\draw [short] (0.75,24) -- (0.75,26.25);
\draw [short] (0.75,26.25) -- (3.25,26.25);
\draw [short] (3.25,26.25) -- (3.25,24);
\draw [short] (0.75,24) -- (3.25,24);
\draw [short] (1.25,26.25) -- (0.75,25.75);
\draw [short] (1.75,26.25) -- (0.75,25.25);
\draw [short] (2.25,26.25) -- (0.75,24.75);
\draw [short] (3,26.25) -- (0.75,24);
\draw [short] (3.25,26) -- (1.25,24);
\draw [short] (3.25,25.5) -- (1.75,24);
\draw [short] (3.25,25) -- (2.25,24);
\draw [short] (3.25,24.5) -- (2.75,24);
\draw [short] (2.75,26.25) -- (2.5,26.25);
\draw [short] (1.75,24) -- (1.75,23.75);
\draw [short] (2.5,24) -- (2.5,23.75);
\draw [short] (-1.25,17) -- (5.75,17);
\draw [short] (-0.75,17) -- (-1,16.25);
\draw [short] (0,17) -- (-0.25,16.25);
\draw [short] (0.75,17) -- (0.5,16.25);
\draw [short] (1.5,17) -- (1.25,16.25);
\draw [short] (2.25,17) -- (2,16.25);
\draw [short] (3,17) -- (2.75,16.25);
\draw [short] (3.75,17) -- (3.5,16.25);
\draw [short] (4.5,17) -- (4.25,16.25);
\draw [short] (5.25,17) -- (5,16.25);
\draw [dashed] (2,28) -- (2.25,14.5);
\draw  (4.75,23) circle (0.75cm);
\draw  (4.75,20.75) circle (0.75cm);
\draw [->, >=Stealth] (4.5,20.5) -- (5,21.25);
\draw [->, >=Stealth] (4.5,22.75) -- (5,23.5);
\draw [->, >=Stealth] (6.5,22) -- (5.5,22.75);
\draw [->, >=Stealth] (6.5,21.25) -- (5.5,20.75);
\draw [->, >=Stealth] (-0.75,24) -- (0.5,24.75);
\draw [->, >=Stealth] (0,21.75) -- (1,21.75);
\node [font=\Large] at (-1.25,24) {drill};
\node [font=\Large] at (-1.25,23.5) {spindle};
\node [font=\Large] at (-0.5,17.75) {drill};
\node [font=\Large] at (-0.5,17.25) {table};
\node [font=\Large] at (6.5,21.5) {sensor};
\node [font=\Large] at (6.25,22.75) {$P$};
\node [font=\Large] at (6.25,20.75) {$Q$};
\node [font=\Large] at (-0.75,21.75) {mandrel};
\draw [->, >=Stealth] (4,23) -- (2.75,23);
\draw [->, >=Stealth] (4,20.75) -- (2.75,20.75);
\draw [short] (1.75,25.25) -- (1.25,24);
\draw [short] (1.75,25.25) -- (2.25,25.25);
\draw [short] (2.25,25.25) -- (2.5,24);
\draw [->, >=Stealth] (2,18.75) .. controls (2.75,18.75) and (2.75,18.25) .. (2,18.25) ;
\end{circuitikz}
}
\label{fig:my_label}
\end{figure}

    \end{enumerate}
    \item In a company, $35\%$ of the employees drink coffee, $40\%$ of the employees
drink tea and $10\%$ of the employees drink both tea and coffee. What $\%$ of
employees drink neither tea nor coffee? \hfill [2021 CE]
\begin{enumerate}
    \begin{multicols}{4}
        \item 15
        \item 25
        \item 35
        \item 40
    \end{multicols}
\end{enumerate}
\item $\oplus $ and $\odot$ are two operators on numbers $p$ and $q$ that $p \oplus q = \frac{p^2 + q^2}{pq}$ and $p \odot q = \frac{p^2}{q}$;
If $x \oplus y = 2$, then $x = $ \hfill [2021 CE]
\begin{enumerate}
    \begin{multicols}{4}
        \item $\frac{y}{2}$
        \item $y$
        \item $\frac{3y}{2}$
        \item $2y$
    \end{multicols}
\end{enumerate}
\item Four persons $P, Q, R$ and $S$ are to be seated in a row, all facing the same
direction, but not necessarily in the same order. $P$ and $R$ cannot sit adjacent
to each other. $S$ should be seated to the right of $Q$. The number of distinct
seating arrangements possible is: \hfill [2021 CE]
\begin{enumerate}
    \begin{multicols}{4}
        \item 2
        \item 4
        \item 6
        \item 8
    \end{multicols}
\end{enumerate}
\item Statement: Either $P$ marries $Q$ or $X$ marries $Y$\\
Among the options below, the logical \textbf{NEGATION} of the above statement is:

\hfill [2021 CE]
\begin{enumerate}
    \item $P$ does not marry $Q$ and $X$ marries $Y$.
    \item Neither $P$ marries $Q$ nor $X$ marries $Y$.
    \item $X$ does not marry $Y$ and $P$ marries $Q$.
    \item $P$ marries $Q$ and $X$ marries $Y$.
\end{enumerate}
\item Consider two rectangular sheets, Sheet $M$ and Sheet $N $ of dimensions
$6 cm \times 4 cm $ each.\\
Folding operation 1: The sheet is folded into half by joining the short edges of
the current shape. \\
Folding operation 2: The sheet is folded into half by joining the long edges of
the current shape.\\
Folding operation 1 is carried out on Sheet $M$ three times.\\
Folding operation 2 is carried out on Sheet $N$ three times. \\
The ratio of perimeters of the final folded shape of Sheet $N$ to the final folded
shape of Sheet $M$ is \rule{2cm}{0.4pt} \hfill [2021 CE]
\begin{enumerate}
    \begin{multicols}{2}
        \item $13 : 7$ \\
        \item $3 :2$ \\
        \item $7 : 5$ \\
        \item $5 : 13$
    \end{multicols}
\end{enumerate}
\item Five line segments of equal lengths, $PR, PS, QS, QT$ and $RT$ are used to
form a star as shown in the figure below.
The value of $\theta$, in degrees, is \rule{2cm}{0.4pt}

\hfill [2021 CE]
\begin{figure}[!ht]
\centering
\resizebox{6cm}{!}{%
\begin{circuitikz}
\tikzstyle{every node}=[font=\normalsize]
\draw [line width=0.5pt, short] (3,12.5) .. controls (2.5,14.5) and (3,14.75) .. (4.25,14.25);
\draw [line width=0.5pt, short] (2.75,12.5) .. controls (2.25,13.75) and (2.25,15.25) .. (4,14.75);
\draw [line width=0.5pt, short] (4,14.75) -- (6.5,14);
\draw [line width=0.5pt, short] (4.25,14.25) -- (6.5,13.5);
\draw [line width=0.5pt, short] (6.5,13.5) .. controls (8.25,13.5) and (8.75,13.75) .. (8.75,14.5);
\draw [line width=0.5pt, short] (6.5,14) .. controls (8,13.75) and (8.5,14) .. (8.5,14.75);
\draw [line width=0.5pt, short] (8.75,14.5) .. controls (9,15.25) and (8.75,15.5) .. (8,16.25);
\draw [line width=0.5pt, short] (8.5,14.75) .. controls (8.25,15.5) and (8.25,15.75) .. (7.5,16);
\draw [line width=0.5pt, short] (7.5,16) .. controls (7.75,16) and (8,16) .. (8,16.25);
\draw [line width=0.5pt, short] (7.5,16) .. controls (7.75,16.25) and (7.75,16.25) .. (8,16.25);
\draw [line width=0.5pt, ->, >=Stealth, dashed] (8,16.25) -- (8.5,18);
\draw [line width=0.5pt, ->, >=Stealth, dashed] (8,16.25) .. controls (7,17) and (6.5,17) .. (6.25,18.25);
\draw [line width=0.5pt, ->, >=Stealth, dashed] (7.75,16.25) -- (5.25,17);
\draw [line width=0.5pt, ->, >=Stealth, dashed] (7.75,16) .. controls (6.25,16.25) and (6.25,16.25) .. (5.25,15.75);
\node [font=\normalsize] at (8.25,16.25) {$E$};
\node [font=\normalsize] at (6.75,18) {$p$};
\node [font=\normalsize] at (8.75,18) {$q$};
\node [font=\normalsize] at (6,17.25) {$n$};
\node [font=\normalsize] at (6,15.75) {$m$};
\end{circuitikz}
}%

\label{fig:my_label}
\end{figure}


\hfill [2021 CE]
\begin{enumerate}
    \begin{multicols}{2}
        \item 36
        \item 45 
        \item 72
        \item 108
    \end{multicols}
\end{enumerate}
\item A function, $\lambda$, is defined by
\begin{center}
    $\lambda\brak{p,q} =
    \begin{cases}
        \brak{p-q}^2 & \text{ if } p \geq q \\
        p + q, & \text{ if } p < q
    \end{cases}$
\end{center}
The value of the expression $\frac{\lambda\brak{-\brak{-3+2}, \brak{-2 + 3}}}{\brak{-\brak{-2 +1}}}$ is: \hfill [2021 CE]
\begin{enumerate}
    \begin{multicols}{2}
    \item $-1$
        \item 0
        \item $\frac{16}{3}$
        \item 16
    \end{multicols}
\end{enumerate}
\item Humans have the ability to construct worlds entirely in their minds, which
don't exist in the physical world. So far as we know, no other species
possesses this ability. This skill is so important that we have different words
to refer to its different flavors, such as imagination, invention and
innovation.\\
Based on the above passage, which one of the following is \textbf{TRUE}? \hfill [2021 CE]
\begin{enumerate}
    \item No species possess the ability to construct worlds in their minds.
    \item The terms imagination, invention and innovation refer to unrelated skills.
    \item We do not know of any species other than humans who possess the ability to
construct mental worlds.
\item Imagination, invention and innovation are unrelated to the ability to construct
mental worlds.
\end{enumerate}
\item The rank of the matrix $\myvec{1 & 2 & 2 & 3 \\ 3 & 4 & 2 & 5 \\ 5 & 6 & 2 & 7 \\ 7 & 8 & 2 & 9}$ is \hfill [2021 CE]
\begin{enumerate}
    \begin{multicols}{4}
        \item 1
        \item 2
        \item 3
        \item 4
    \end{multicols}
\end{enumerate}
\item If $P = \myvec{1 & 2 \\ 3 &4}$ and $Q = \myvec{0 & 1 \\ 1 & 0}$ then $Q^TP^T$ is \hfill [2021 CE]
\begin{enumerate}
\begin{multicols}{4}
    \item $\myvec{1 & 2 \\ 3 & 4}$
     \item $\myvec{1 & 3 \\ 2 & 4}$
      \item $\myvec{2 & 1 \\ 4& 3}$
       \item $\myvec{2 & 4 \\ 1 & 3}$
\end{multicols}
\end{enumerate}
\item The shape of the cumulative distribution function of Gaussian distribution is 

\hfill [ 2021 CE]
\begin{enumerate}
    \item Horizontal line
    \item Straight line at 45 degree angle
    \item Bell-shaped
    \item S-shaped
\end{enumerate}
\end{enumerate}
\end{document}
