\let\negmedspace\undefined
\let\negthickspace\undefined
\documentclass[journal]{IEEEtran}
\usepackage[a5paper, margin=10mm, onecolumn]{geometry}
\usepackage{lmodern} % Ensure lmodern is loaded for pdflatex
\usepackage{tfrupee} % Include tfrupee package

\setlength{\headheight}{1cm} % Set the height of the header box
\setlength{\headsep}{0mm}     % Set the distance between the header box and the top of the text

\usepackage{gvv-book}
\usepackage{gvv}
\usepackage{cite}
\usepackage{amsmath,amssymb,amsfonts,amsthm}
\usepackage{algorithmic}
\usepackage{graphicx}
\usepackage{textcomp}
\usepackage{xcolor}
\usepackage{txfonts}
\usepackage{listings}
\usepackage{enumitem}
\usepackage{mathtools}
\usepackage{gensymb}
\usepackage{comment}
\usepackage[breaklinks=true]{hyperref}
\usepackage{tkz-euclide} 
\usepackage{listings}
% \usepackage{gvv}                                        
\def\inputGnumericTable{}                                 
\usepackage[latin1]{inputenc}                                
\usepackage{color}                                            
\usepackage{array}                                            
\usepackage{longtable}                                       
\usepackage{calc}                                             
\usepackage{multirow}                                         
\usepackage{hhline}                                           
\usepackage{ifthen}                                           
\usepackage{lscape}
\begin{document}

\bibliographystyle{IEEEtran}
\vspace{3cm}

\title{2021-AE-1-13}
\author{EE24BTECH11010 - BALAJI B}
% \maketitle
% \newpage
% \bigskip
{\let\newpage\relax\maketitle}

\renewcommand{\thefigure}{\theenumi}
\renewcommand{\thetable}{\theenumi}
\setlength{\intextsep}{10pt} % Space between text and floats


\numberwithin{equation}{enumi}
\numberwithin{figure}{enumi}
\renewcommand{\thetable}{\theenumi}
\begin{enumerate}
    \item (i) Arun and Aparna are here.\\
    (ii) Arun and Aparna is here. \\
    (iii) Arun's families is here. \\
    (iv) Arun's family is here. \\
    Which of the above sentences are grammatically \textbf{CORRECT}? \hfill[2021 AE]
    \begin{enumerate}
        \begin{multicols}{4}
            \item (i) and (ii)
            \item (i) and (iv)
            \item (ii) and (iv)
            \item (iii) and (iv)
        \end{multicols}
    \end{enumerate}
    \item The mirror image of the below text about the $x$-axis is \hfill[2021 AE]
     \begin{figure}[H]
\centering
\resizebox{10cm}{!}{%
\begin{circuitikz}
\tikzstyle{every node}=[font=\normalsize]
\draw (6,14) to[R] (8,14);
\draw (4.5,14) to[L ] (6.5,14);
\draw (8,14) to[short] (9,14);
\draw (9,14) to[short] (9,13.5);
\draw (9,13.5) to[european resistor] (9,12);
\draw (9,12) to[short] (4.5,12);
\draw (4.5,14.5) to[short] (4.5,11.75);
\draw (4.5,14.5) to[short] (1.75,14.5);
\draw (1.75,14.5) to[short] (1.75,11.75);
\draw (1.75,11.75) to[short] (4.5,11.75);
\draw [short] (3,14) -- (3,13.5);
\draw [short] (2.5,13.5) -- (3.5,13.5);
\draw [short] (3,13.5) -- (2.5,13);
\draw [short] (3,13.5) -- (3.5,13);
\draw [short] (2.5,13) -- (3.5,13);
\draw [short] (3,13) -- (3,12.5);
\draw [short] (3.25,13.5) -- (3.5,13.75);
\node [font=\normalsize] at (5.5,14.75) {Filter};
\node [font=\normalsize] at (7,14.75) {Choke};
\node [font=\normalsize] at (10,12.75) {Battery};
\draw [short] (-2,14.75) -- (-2,11.75);
\draw [short] (-2.5,14.75) -- (-2.5,11.75);
\draw (-3.25,13.25) to[tmultiwire] (-6.75,13.25);
\draw (1.75,13.25) to[tmultiwire] (-1.25,13.25);
\draw  (-7.5,13.25) circle (0.75cm);
\draw [short] (-8,13.25) .. controls (-7.5,13.75) and (-7.5,12.5) .. (-7,13.25);
\draw [decorate, decoration={coil, aspect=0.4, segment length=6pt, amplitude=10pt}] (-1.5,12) -- (-1.5,14.5);
\draw [decorate, decoration={coil, aspect=0.4, segment length=6pt, amplitude=10pt}] (-3,12) -- (-3,14.5);
\end{circuitikz}
}%

\label{fig:my_label}
\end{figure} 
    \begin{enumerate}
        \item \begin{figure}[!ht]
\centering
\resizebox{6.5cm}{!}{%
\begin{circuitikz}
\tikzstyle{every node}=[font=\large]
\draw [line width=0.7pt, short] (3.25,13.25) -- (10.25,11.5);
\draw [line width=0.7pt, short] (3.25,13.25) -- (3.5,13.5);
\draw [line width=0.7pt, short] (3.5,13.5) -- (10.5,11.75);
\draw [line width=0.7pt, short] (10.5,11.75) -- (10.25,11.5);
\draw [line width=0.7pt, ->, >=Stealth] (3.5,13.5) -- (4.5,14.25);
\draw [line width=0.7pt, ->, >=Stealth] (10.5,11.75) -- (11.75,13);
\node [font=\large] at (3.25,12.75) {$E$};
\node [font=\large] at (10,11.25) {$F$};
\node [font=\large] at (4,14.25) {$V_E$};
\node [font=\large] at (12,12.5) {$V_F$};
\end{circuitikz}
}%

\label{fig:my_label}
\end{figure}
        \item \begin{figure}[!ht]
\centering
\resizebox{7cm}{!}{%
\begin{circuitikz}
\tikzstyle{every node}=[font=\large]
\draw [ line width=0.7pt](-3,15.75) to[american voltage source] (-3,13);
\draw [ line width=0.7pt](-3,15.75) to[short] (-1.5,15.75);
\draw [ line width=0.7pt](-3,13) to[short] (-1.5,13);
\draw [ line width=0.7pt](-1.5,16.5) to[short] (-1.5,12.25);
\draw [ line width=0.7pt](-1.5,16.5) to[short] (2.25,16.5);
\draw [ line width=0.7pt](2.25,16.5) to[short] (2.25,12.25);
\draw [ line width=0.7pt](-1.5,12.25) to[short] (2.25,12.25);
\draw [ line width=0.7pt](2.25,13.25) to[short] (5.25,13.25);
\draw [line width=0.7pt](5.25,13.25) to[C] (5.25,15.75);
\draw [line width=0.7pt](2.25,15.75) to[L ] (5.25,15.75);
\draw [ line width=0.7pt](5.25,15.75) to[short] (6.75,15.75);
\draw [ line width=0.7pt](5.25,13.25) to[short] (6.75,13.25);
\draw [ line width=0.7pt](6.75,15.75) to[R] (6.75,13.25);
\node [font=\large] at (-3.75,14.25) {$V_{in}$};
\node [font=\large] at (0.5,14.5) {Full-Bridge};
\node [font=\large] at (0.25,13.75) {VSI};
\node [font=\large] at (2.5,14.5) {$V_R$};
\node [font=\large] at (2.5,16) {$+$};
\node [font=\LARGE] at (2.5,13) {$-$};
\node [font=\large] at (4.5,14.5) {$C$};
\node [font=\large] at (6.25,14.5) {$R$};
\node [font=\large] at (7.25,14.5) {$V_o$};
\node [font=\large] at (7,15.5) {$+$};
\node [font=\LARGE] at (7,13.25) {$-$};
\node [font=\large] at (3.75,16.5) {$L$};
\end{circuitikz}
}%

\label{fig:my_label}
\end{figure}

        \item \begin{figure}[H]
\centering
\resizebox{2.5cm}{!}{%
\begin{circuitikz}
\tikzstyle{every node}=[font=\Large]
\node [font=\Large, rotate around={180:(0,0)}] at (1.75,14.25) {T};
\node [font=\Large, rotate around={180:(0,0)}] at (2.15,14.25) {R};
\node [font=\Large] at (2.5,14.25) {I};
\node [font=\Large, rotate around={180:(0,0)}] at (2.75,14.25) {A};
\node [font=\Large, rotate around={-180:(0,0)}] at (3.25,14.25) {N};
\node [scale=-1, yscale=-1, font=\Large, rotate around={180:(0,0)}] at (3.75,14.25) {G};
\node [font=\Large, rotate around={180:(0,0)}] at (4.15,14.25) {L};
\node [font=\Large] at (4.5,14.25) {E};
\end{circuitikz}
}%

\label{fig:my_label}
\end{figure}

        \item \begin{figure}[H]
\centering
\resizebox{4cm}{!}{%
\begin{circuitikz}
\tikzstyle{every node}=[font=\large]
\draw [short] (1.25,23.75) -- (1.25,19.75);
\draw [short] (1.25,23.75) -- (2.75,23.75);
\draw [short] (2.75,23.75) -- (2.75,19.75);
\draw [short] (1.25,19.75) -- (2.75,19.75);
\draw [short] (0.75,24) -- (0.75,26.25);
\draw [short] (0.75,26.25) -- (3.25,26.25);
\draw [short] (3.25,26.25) -- (3.25,24);
\draw [short] (0.75,24) -- (3.25,24);
\draw [short] (1.25,26.25) -- (0.75,25.75);
\draw [short] (1.75,26.25) -- (0.75,25.25);
\draw [short] (2.25,26.25) -- (0.75,24.75);
\draw [short] (3,26.25) -- (0.75,24);
\draw [short] (3.25,26) -- (1.25,24);
\draw [short] (3.25,25.5) -- (1.75,24);
\draw [short] (3.25,25) -- (2.25,24);
\draw [short] (3.25,24.5) -- (2.75,24);
\draw [short] (2.75,26.25) -- (2.5,26.25);
\draw [short] (1.75,24) -- (1.75,23.75);
\draw [short] (2.5,24) -- (2.5,23.75);
\draw [short] (-1.25,17) -- (5.75,17);
\draw [short] (-0.75,17) -- (-1,16.25);
\draw [short] (0,17) -- (-0.25,16.25);
\draw [short] (0.75,17) -- (0.5,16.25);
\draw [short] (1.5,17) -- (1.25,16.25);
\draw [short] (2.25,17) -- (2,16.25);
\draw [short] (3,17) -- (2.75,16.25);
\draw [short] (3.75,17) -- (3.5,16.25);
\draw [short] (4.5,17) -- (4.25,16.25);
\draw [short] (5.25,17) -- (5,16.25);
\draw [dashed] (2,28) -- (2.25,14.5);
\draw  (4.75,23) circle (0.75cm);
\draw  (4.75,20.75) circle (0.75cm);
\draw [->, >=Stealth] (4.5,20.5) -- (5,21.25);
\draw [->, >=Stealth] (4.5,22.75) -- (5,23.5);
\draw [->, >=Stealth] (6.5,22) -- (5.5,22.75);
\draw [->, >=Stealth] (6.5,21.25) -- (5.5,20.75);
\draw [->, >=Stealth] (-0.75,24) -- (0.5,24.75);
\draw [->, >=Stealth] (0,21.75) -- (1,21.75);
\node [font=\Large] at (-1.25,24) {drill};
\node [font=\Large] at (-1.25,23.5) {spindle};
\node [font=\Large] at (-0.5,17.75) {drill};
\node [font=\Large] at (-0.5,17.25) {table};
\node [font=\Large] at (6.5,21.5) {sensor};
\node [font=\Large] at (6.25,22.75) {$P$};
\node [font=\Large] at (6.25,20.75) {$Q$};
\node [font=\Large] at (-0.75,21.75) {mandrel};
\draw [->, >=Stealth] (4,23) -- (2.75,23);
\draw [->, >=Stealth] (4,20.75) -- (2.75,20.75);
\draw [short] (1.75,25.25) -- (1.25,24);
\draw [short] (1.75,25.25) -- (2.25,25.25);
\draw [short] (2.25,25.25) -- (2.5,24);
\draw [->, >=Stealth] (2,18.75) .. controls (2.75,18.75) and (2.75,18.25) .. (2,18.25) ;
\end{circuitikz}
}
\label{fig:my_label}
\end{figure}

    \end{enumerate}
    
    \item Two identical cube shaped dice each with faces numbered 1 to 6 are rolled
simultaneously. The probability that an even number is rolled out on each
dice is:

\hfill [2021 AE]
\begin{enumerate}
    \begin{multicols}{4}
        \item $\frac{1}{36}$
        \item $\frac{1}{12}$
        \item $\frac{1}{8}$
        \item $\frac{1}{4}$
    \end{multicols}
\end{enumerate}
\item $\oplus$ and $\odot$ are two operators on numbers $p$ and $q$ such that $p \odot q = p - q$, and $p \oplus q = p \times q$. Then, $\brak{9 \odot \brak{6 \oplus 7}} \odot \brak{7 \oplus \brak{6 \odot 5}} = $ \hfill[2021 AE] 
\begin{enumerate}
    \begin{multicols}{2}
        \item 40
        \item $-26$
        \item $-33$
        \item $-40$
    \end{multicols}
\end{enumerate}
\item Four persons $P, Q, R$ and $S$ are to be seated in a row. $R$ should not be seated
at the second position from the left end of the row. The number of distinct
seating arrangements possible is: \hfill [2021 AE]
\begin{enumerate}
    \begin{multicols}{4}
        \item 6
        \item 9
        \item 18
        \item 24
    \end{multicols}
\end{enumerate}
\item On a planar field, you travelled 3 units East from a point $O$. Next you
travelled 4 units South to arrive at point $ P$. Then you travelled from $P$ in the
North-East direction such that you arrive at a point that is 6 units East of
point $O$. Next, you travelled in the North-West direction, so that you arrive
at point $Q$ that is 8 units North of point $P$.
The distance of point $ Q $ to point $ O$, in the same units, should be \rule{2cm}{0.4pt}

\hfill [2021 AE]
\begin{enumerate}
    \begin{multicols}{4}
        \item 3
        \item 4
        \item 5
        \item 6
    \end{multicols}
\end{enumerate}
\item The author said, "Musicians rehearse before their concerts. Actors rehearse
their roles before the opening of a new play. On the other hand, I find it
strange that many public speakers think they can just walk on to the stage
and start speaking. In my opinion, it is no less important for public speakers
to rehearse their talks." \\ \\ Based on the above passage, which one of the following is \textbf{TRUE}? \hfill[2021 AE]
\begin{enumerate}
    \item The author is of the opinion that rehearsing is important for musicians, actors
and public speakers.
\item The author is of the opinion that rehearsing is less important for public speakers
than for musicians and actors.
\item The author is of the opinion that rehearsing is more important only for
musicians than public speakers.
\item The author is of the opinion that rehearsal is more important for actors than
musicians.
\end{enumerate}
\item 1. Some football players play cricket.\\
2. All cricket players play hockey. \\ 
Among the options given below, the statement that logically follows from the two statements 1 and 2 above, is: \hfill [2021 AE]
\begin{enumerate}
    \item No football player plays hockey.
    \item Some football players play hockey.
    \item All football players play hockey.
    \item All hockey players play football.
\end{enumerate}
\item 
In the figure shown above, $PQRS$ is a square. The shaded portion is formed
by the intersection of sectors of circles with radius equal to the side of the
square and centers at $S $and $Q.$ \\
The probability that any point picked randomly within the square falls in the
shaded area is \rule{2cm}{0.4pt} \hfill [2021 AE]
\begin{figure}[!ht]
\centering
\resizebox{6cm}{!}{%
\begin{circuitikz}
\tikzstyle{every node}=[font=\normalsize]
\draw [line width=0.5pt, short] (3,12.5) .. controls (2.5,14.5) and (3,14.75) .. (4.25,14.25);
\draw [line width=0.5pt, short] (2.75,12.5) .. controls (2.25,13.75) and (2.25,15.25) .. (4,14.75);
\draw [line width=0.5pt, short] (4,14.75) -- (6.5,14);
\draw [line width=0.5pt, short] (4.25,14.25) -- (6.5,13.5);
\draw [line width=0.5pt, short] (6.5,13.5) .. controls (8.25,13.5) and (8.75,13.75) .. (8.75,14.5);
\draw [line width=0.5pt, short] (6.5,14) .. controls (8,13.75) and (8.5,14) .. (8.5,14.75);
\draw [line width=0.5pt, short] (8.75,14.5) .. controls (9,15.25) and (8.75,15.5) .. (8,16.25);
\draw [line width=0.5pt, short] (8.5,14.75) .. controls (8.25,15.5) and (8.25,15.75) .. (7.5,16);
\draw [line width=0.5pt, short] (7.5,16) .. controls (7.75,16) and (8,16) .. (8,16.25);
\draw [line width=0.5pt, short] (7.5,16) .. controls (7.75,16.25) and (7.75,16.25) .. (8,16.25);
\draw [line width=0.5pt, ->, >=Stealth, dashed] (8,16.25) -- (8.5,18);
\draw [line width=0.5pt, ->, >=Stealth, dashed] (8,16.25) .. controls (7,17) and (6.5,17) .. (6.25,18.25);
\draw [line width=0.5pt, ->, >=Stealth, dashed] (7.75,16.25) -- (5.25,17);
\draw [line width=0.5pt, ->, >=Stealth, dashed] (7.75,16) .. controls (6.25,16.25) and (6.25,16.25) .. (5.25,15.75);
\node [font=\normalsize] at (8.25,16.25) {$E$};
\node [font=\normalsize] at (6.75,18) {$p$};
\node [font=\normalsize] at (8.75,18) {$q$};
\node [font=\normalsize] at (6,17.25) {$n$};
\node [font=\normalsize] at (6,15.75) {$m$};
\end{circuitikz}
}%

\label{fig:my_label}
\end{figure}

\begin{enumerate}
    \begin{multicols}{2}
        \item $4 - \frac{\pi}{2}$
        \item $\frac{1}{2}$
        \item $\frac{\pi}{2} - 1$
        \item $\frac{\pi}{4}$
    \end{multicols}
\end{enumerate}
\item In an equilateral triangle $PQR$, side $PQ$ is divided into four equal parts, side
$QR$ is divided into six equal parts and side $PR$ is divided into eight equal parts.
The length of each subdivided part in $cm$ is an integer. \\
The minimum area of the triangle $PQR$ possible, in $cm^2$
, is \hfill[2021 AE]
\begin{enumerate}
    \begin{multicols}{4}
        \item 18
        \item 24
        \item $48\sqrt{3}$
        \item $144\sqrt{3}$
    \end{multicols}
\end{enumerate}
\item Consider the differential equation $\frac{d^2y}{dx^2} + 8 \frac{dy}{dx} + 16y = 0$
and the boundary
conditions
$y(0) = 1$
and $\frac{dy}{dx}
(0) = 0$ 
. The solution to this equation is: \hfill [2021 AE]
\begin{enumerate}
\begin{multicols}{2}
\item $y = \brak{1 + 2x}e^{-4x}$
    \item $y = \brak{1 - 4x}e^{-4x}$
   \item $y = \brak{1 + 8x}e^{-4x}$
     \item $y = \brak{1 + 4x}e^{-4x}$
\end{multicols}
    \end{enumerate}
    \item $u \brak{x,y}$ is governed by the following equation $\frac{\partial^2 u}{\partial x^2} -4 \frac{\partial^2 u}{\partial x \partial y} + 6 \frac{\partial^ u}{\partial y^2} = x + 2y
$. The nature of this equation is: \hfill [2021 AE]
\begin{enumerate}
\begin{multicols}{2}
    \item linear
    \item elliptic
    \item hyperbolic
\item parabolic
\end{multicols}
\end{enumerate}
\item Consider the velocity field $\vec{V} = \brak{2x + 3y}\hat{i} + \brak{3x + 2y} \hat{j}$. The field $\vec{V}$ is \hfill [2021 AE]
\begin{enumerate}
    \item divergence-free and curl-free
    \item curl-free but not divergence-free
    \item divergence-free but not curl-free
    \item neither divergence-free nor curl-free
\end{enumerate}

\end{enumerate}

\end{document}
