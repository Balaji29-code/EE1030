\let\negmedspace\undefined
\let\negthickspace\undefined
\documentclass[journal]{IEEEtran}
\usepackage[a5paper, margin=10mm, onecolumn]{geometry}
%\usepackage{lmodern} % Ensure lmodern is loaded for pdflatex
\usepackage{tfrupee} % Include tfrupee package

\setlength{\headheight}{1cm} % Set the height of the header box
\setlength{\headsep}{0mm}     % Set the distance between the header box and the top of the text

\usepackage{gvv-book}
\usepackage{gvv}
\usepackage{cite}
\usepackage{amsmath,amssymb,amsfonts,amsthm}
\usepackage{algorithmic}
\usepackage{graphicx}
\usepackage{textcomp}
\usepackage{xcolor}
\usepackage{txfonts}
\usepackage{listings}
\usepackage{enumitem}
\usepackage{mathtools}
\usepackage{gensymb}
\usepackage{comment}
\usepackage[breaklinks=true]{hyperref}
\usepackage{tkz-euclide} 
\usepackage{listings}
% \usepackage{gvv}                                        
\def\inputGnumericTable{}                                 
\usepackage[latin1]{inputenc}                                
\usepackage{color}                                            
\usepackage{array}                                            
\usepackage{longtable}                                       
\usepackage{calc}                                             
\usepackage{multirow}                                         
\usepackage{hhline}                                           
\usepackage{ifthen}                                           
\usepackage{lscape}
\begin{document}

\bibliographystyle{IEEEtran}
\vspace{3cm}

\title{2024-MA-14-26}
\author{EE24BTECH11010 - BALAJI B}
% \maketitle
% \newpage
% \bigskip
{\let\newpage\relax\maketitle}

\renewcommand{\thefigure}{\theenumi}
\renewcommand{\thetable}{\theenumi}
\setlength{\intextsep}{10pt} % Space between text and floats


\numberwithin{equation}{enumi}
\numberwithin{figure}{enumi}
\renewcommand{\thetable}{\theenumi}
\begin{enumerate}
    \item Consider the following limit 
    \begin{center}
        $\lim\limits_{\epsilon \to 0} \frac{1}{\epsilon} \int_0^\infty e^{-x / \epsilon}\brak{\cos{\brak{3x}} + x^2  \sqrt{x +4}}dx$
    \end{center}
    Which one of the following is correct? \hfill [2024 MA]
    \begin{enumerate}
        \item The limit does not exist
        \item The limit exists and is equal to 0
        \item The limit exists and is equal to 3
        \item The limit exists and is equal to $\pi$
    \end{enumerate}
    \item Let $\mathbb{R}\sbrak{X^2, X^3}$ be a substring of $\mathbb{R} \sbrak{X}$ generated by $X^2$ and $X^3$. Consider the following
statements.\\
I. The ring $\mathbb{R} \sbrak{X^2, X^3}$ is a unique factorization domain.\\
II. The ring $\mathbb{R} \sbrak{X^2, X^3}$ is a principle ideal domain. \\
Which one of the following is correct? \hfill [2024 MA]
\begin{enumerate}
    \item Both I and II are TRUE
    \item I is TRUE and II is FALSE
    \item I is FALSE and II is TRUE
    \item Both I and II are FALSE
\end{enumerate}
\item Given a prime number  $p$, let $n_p\brak{G}$ denote the number of $p$-Sylow subgroups of a finite group $G$. Which one of the following is TRUE for every group $G$ of order 2024? \hfill [2024 MA]
\begin{enumerate}
    \item $n_{11}\brak{G} = 1$ and $ n_{23}\brak{G} = 11$
    \item $n_{11}\brak{G} \in \cbrak{1,23}$ and 
    $n_{23}\brak{G} = 1$
    \item $n_{11}\brak{G} = 23$ and $n_{23} \in \cbrak{1,88}$
    \item $n_{11}\brak{G} = 23$ and $n_{23}\brak{G} = 11$
\end{enumerate}
\item Consider the following statements.\\
I. Every compact Hausdroff space is normal. \\
II. Every metric space is normal.
Which one of the following is correct?

\hfill [2024 MA]
\begin{enumerate}
    \item Both I and II are TRUE 
    \item I is TRUE and II is FALSE 
    \item I is FALSE and II is TRUE 
    \item Both I and II are FALSE
\end{enumerate}
\item Consider the topology on $\mathbb{Z}$ with basis $\cbrak{S\brak{a,b} : a,b \in \mathbb{Z} \text{
and 
} a \neq 0}$, where 
\begin{center}
    $S\brak{a,b} = \cbrak{an + b: n \in \mathbb{Z}} $
\end{center}
Consider the following statements. \\
I. $S\brak{a,b}$ is both open and closed for each $a,b \in \mathbb{Z}$ with $a \neq 0$\\
II. The only connected set containing $x \in \mathbb{Z}$ is $\cbrak{x}$ \\
Which one of the following is correct? \hfill [2024 MA]
\begin{enumerate}
    \item Both I and II are TRUE
    \item I is TRUE and II is FALSE
    \item I is FALSE and I is TRUE
    \item Both I and II are FALSE
\end{enumerate}
\item Let $A \in  M_2\brak{\mathbb{C}}$ be a normal matrix. Consider the following statements.\\
I. If all the eigenvalues of $A$ are real, then $A$ is Hermitian.\\
II. If all the eigenvalues of $A$ have absolute value 1, then $A$ is unitary.\\
Which one of the following is correct? \hfill [2024 MA]
\begin{enumerate}
    \item Both I and II are TRUE
    \item I is TRUE and II is FALSE
    \item I is FALSE and II is TRUE
    \item Both I and II are FALSE
\end{enumerate}
\item Let $A = \myvec{0 & 2 \\ 2 & 0}$ and $T: M_2\brak{\mathbb{C}} \to M_2 \brak{\mathbb{C}}$ be a linear transformation given by $T\brak{B} = AB$. The characteristic polynomial of $T$ is \hfill [2024 MA]
\begin{enumerate}
\begin{multicols}{2}
    \item $X^2 - 8X + 16$
    \item $X^2 - 4$
    \item $X^2 - 2$
    \item $X^2 - 16$
\end{multicols}
\end{enumerate}
\item Let 
\begin{center}
    $A = \myvec{2 & -1 & 1 \\ 1 & 2 & -1 \\
    -1 & 1 & 2}$
\end{center}
and $\vec{b}$ be a $3 \times 1$ real column vector. Consider the statements. \\
I. The Jacobi iteration method for the system $\brak{A + \epsilon I_3} \vec{x} = \vec{b}$ onverges for any
initial approximation and $\epsilon > 0$ \\
II. The Gauss-Seidel iteration method for the system $\brak{A + \epsilon I_3} \vec{x} = \vec{b}$ converges for any initial approximation and $\epsilon > 0$ \\
Which of the following is correct? \hfill [2024 MA]
\begin{enumerate}
    \item Both I and I are TRUE
    \item I is TRUE and II is FALSE
    \item I is FALSE and II is TRUE
    \item Both I and I are FALSE
\end{enumerate}
\item For the initial value problem 
\begin{center}
    $y^\prime = f \brak{x,y}$,  $y \brak{x_0} = y_0$
\end{center}
generate approximations $y_n$ to $y\brak{x_n}, x_n = x_0 + nh$, for a fixed $h > 0$ and $n = 1,2,3, \dots$ using the recursion formula 
\begin{center}
    $y_n = y_{n-1} + ak_1 + bk_2$, where \\
    $k_1 = hf\brak{x_{n-1}, y_{n-1}}$ and $k_2 = hf\brak{x_{n-1} + \alpha h, y_{n-1} + \beta k_1}$
\end{center}
Which one of the following choices of $a,b, \alpha, \beta$ for the above recursion formula gives the Runge-Kuta method of order 2? \hfill [2024 MA]
\begin{enumerate}
    \item $a = 1, b = 1, \alpha = 0.5, \beta = 0.5 $
     \item $a = 0.5, b = 0.5, \alpha = 2, \beta = 2 $
      \item $a = 0.25, b = 0.75, \alpha = \frac{2}{3}, \beta = \frac{2}{3}$
       \item $a = 0.5, b = 0.5, \alpha = 1, \beta = 2 $
\end{enumerate}
\item Let $u = u\brak{x,t}$ be the solution of 
\begin{center}
    $\frac{\partial u}{\partial t} - 4 \frac{\partial^2 u}{\partial x^2} = 0$ , $0 < x < 1$ , $t > 0$, \\
    $u\brak{0,t} = u\brak{1,t} = 0$ , $ t \geq 0$, \\
    $u\brak{x,0} = \sin \brak{\pi x}$ , $ 0 \leq x \leq 1$ 
\end{center}
Define $g\brak{t} = \int \limits_0^1 \brak{u \brak{x,t}^2} dx$, for $t > 0$. Which one of the following is correct?  

\hfill [2024 MA]
\begin{enumerate}
    \item $g$ is decreasing on $\brak{0 , \infty}$ and $\lim \limits_{t \to \infty} g \brak{t} = 0$
    \item $g$ is decreasing on $\brak{0 , \infty}$ and $\lim \limits_{t \to \infty} g \brak{t} = \frac{1}{4}$
    \item $g$ is increasing on $\brak{0 , \infty}$ and $\lim \limits_{t \to \infty} g \brak{t} $ does not exist 
    \item $g$ is increasing on $\brak{0 , \infty}$ and $\lim \limits_{t \to \infty} g \brak{t} = 3$
\end{enumerate}
\item  $y_1$ and $y_2$ are two different solution of the ordinary differential equation 
\begin{center}
    $y^\prime + \sin \brak{e^x} y = \cos \brak{e^{x+1}}$ , $ 0 \leq x \leq 1$, 
\end{center}
then which of the following is the general solution on $\sbrak{0,1}$? \hfill [2024 MA]
\begin{enumerate}
    \begin{multicols}{2}
        \item $c_1 y_1 + c_2 + y_2$ , $c_1, c_2 \in \mathbb{R}$
        \item $y_1 = c_1 \brak{y_1 - y_2}$ , $c \in \mathbb{R}$
        \item $cy_1 + \brak{y_1 - y_2}$ , $c \in \mathbb{R}$
        \item $c_1 \brak{y_1 + y_2} + c_2 \brak{y_1 - y_2}$ , $c_1 , c_2 \in \mathbb{R}$
    \end{multicols}
\end{enumerate}
\item Consider the following Linear Programming Problem \textbf{P}
\begin{table}[H]
    \centering
    \begin{tabular}{c c}
minimize & $5x_1 + 2x_2$\\
subject to & $2x_1 + x_2 \leq 2,$ \\
 & $x_1 + x_2 \geq 1$\\
 & $x_1 , x_2 \geq 0$.
\end{tabular}
\end{table}
The optimal value of the problem \textbf{P} is equal to  \hfill [2024 MA]
\begin{enumerate}
    \begin{multicols}{4}
        \item 5
        \item 0 
        \item 4 
        \item 2
    \end{multicols}
\end{enumerate}
 \item Let $p = \brak{1 , \frac{1}{2} , \frac{1}{3}, \frac{1}{4}} \in \mathbb{R}^4$ and $f : \mathbb{R}^4 \to \mathbb{R}$ br a differential function such that $f\brak{p} = 6$ and $f \brak{\lambda x} = \lambda ^3 f\brak{x}$, for every $\lambda \in \brak{0, \infty}$ and $x \in \mathbb{R}^4$. The value of 
 \begin{center}
     $12 \frac{\partial f}{\partial x_1} \brak{p} + 6 \frac{\partial f}{\partial x_2} \brak{p} + 4 \frac{\partial f}{ \partial x_3} \brak{p} + 3 \frac{\partial f}{\partial x_4}\brak{p}$
 \end{center}
 is equal to \rule{2cm}{0.4pt} (answer in integer) \hfill [2024 MA]
\end{enumerate}
\end{document} 
