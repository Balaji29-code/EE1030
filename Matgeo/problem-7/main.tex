\let\negmedspace\undefined
\let\negthickspace\undefined
\documentclass[journal]{IEEEtran}
\usepackage[a5paper, margin=10mm, onecolumn]{geometry}
\usepackage{lmodern} % Ensure lmodern is loaded for pdflatex
\usepackage{tfrupee} % Include tfrupee package

\setlength{\headheight}{1cm} % Set the height of the header box
\setlength{\headsep}{0mm}     % Set the distance between the header box and the top of the text

\usepackage{gvv-book}
\usepackage{gvv}
\usepackage{cite}
\usepackage{amsmath,amssymb,amsfonts,amsthm}
\usepackage{algorithmic}
\usepackage{graphicx}
\usepackage{textcomp}
\usepackage{xcolor}
\usepackage{txfonts}
\usepackage{listings}
\usepackage{mathtools}
\usepackage{gensymb}
\usepackage{enumitem}
\usepackage{comment}
\usepackage[breaklinks=true]{hyperref}
\usepackage{tkz-euclide} 
\usepackage{listings}
\usepackage{gvv}                                        
\def\inputGnumericTable{}                                 
\usepackage[latin1]{inputenc}                                
\usepackage{color}                                            
\usepackage{array}                                            
\usepackage{longtable}                                       
\usepackage{calc}                                             
\usepackage{multirow}                                         
\usepackage{hhline}                                           
\usepackage{ifthen}                                           
\usepackage{lscape}
\begin{document}

\bibliographystyle{IEEEtran}
\vspace{3cm}

\title{9.9.3.9}
\author{EE24BTECH11010 - Balaji B}
% \maketitle
% \newpage
% \bigskip
{\let\newpage\relax\maketitle}
\textbf{Question: } \\ 
If the area of the region bounded by the curve $y^2 = 4ax$ and the line $x = 4a$ is $\frac{256}{3}$ sq. units, then using integration, find the value of $a$, where $a > 0$. 

\textbf{Solution:}
\begin{table}[h!]    
  \centering
  
\begin{tabular}[12pt]{ |c|c|}
    \hline
    \textbf{Parameter} & \textbf{Description}\\ 
    \hline
    $\vec{x_1}$ & First intersection point\\
    \hline
    $\vec{x_2}$ & Second intersection point\\
    \hline
    $\vec{h}$ & Point on the given line\\
    \hline
    $\vec{m}$ & Direction vector of given line\\
    \hline
    $A$ & Area of the region\\
    \hline
    \end{tabular}


  \caption{Variables Used}
  \label{tab1-1.9-6}
\end{table}\\
The equation of a parabola in Matrix form is
\begin{align}
\vec{x}^\top\vec{V}\vec{x} + 2\vec{u}^\top\vec{x} + f = 0
\end{align}
The equation of a line in vector form is
\begin{align}
\vec{x}&=\vec{h}+\kappa\vec{m}
\end{align}
For the given parabola $y^2=4ax$, The values of $\vec{V}$,$\vec{u}$,$f$ are
\begin{align}
\vec{V}&=\myvec{0 & 0\\0 & 1}\\
\vec{u}&=\myvec{-2a\\0}\\
f&=0
\end{align}
For the given line $x=4a$, The values of $\vec{h}$, $\vec{m}$ are
\begin{align}
\vec{h}&=\myvec{4a\\0}\\
\vec{m}&=\myvec{0\\1}
\end{align}
Substituting the line equation in parabola equation gives the values of $\kappa$
\begin{align}
\brak{\vec{h}+\kappa\vec{m}}^\top\vec{V}\brak{\vec{h}+\kappa\vec{m}} + 2\vec{u}^\top\brak{\vec{h}+\kappa\vec{m}} + f &= 0\\
\brak{\myvec{4a\\0}+\kappa\myvec{0\\1}}^\top\myvec{0 & 0\\0 & 1}\brak{\myvec{4a\\0}+\kappa\myvec{0\\1}} + 2\myvec{-2a\\0}^\top\brak{\myvec{4a\\0}+\kappa\myvec{0\\1}} + 0 &= 0\\
\myvec{4a&\kappa}\myvec{0&0\\0&1}\myvec{4a\\\kappa}+2\myvec{-2a&0}\myvec{4a\\\kappa} &= 0\\
\myvec{4a&\kappa}\myvec{0\\\kappa}+2\brak{-8a^2} &= 0\\
\kappa^2-16a^2&= 0\\
\kappa_1&=4a\\
\kappa_2&=-4a\\
\end{align}
The intersection points are
\begin{align}
\vec{x_1} &= \vec{h}+\kappa_1\vec{m}\\
\vec{x_1} &= \myvec{4a\\4a}\\
\vec{x_2} &= \vec{h}+\kappa_2\vec{m}\\
\vec{x_2} &= \myvec{4a\\-4a}
\end{align}
The Area under the curve is given by
\begin{align}
A &= \int_{-4a}^{4a}\brak{\frac{y^2}{4a}}dy\\
A &= \brak{\frac{1}{4a}}\brak{\frac{(4a)^3 - \brak{(-4a)^3}}{3}}\\
A &= \brak{\frac{1}{12a}}\brak{128a^3}\\
A &= \frac{32a^2}{3}
\end{align}
The area of region bounded by the line $x=4a$ and the parabola $y^2=4ax$ is given as $\frac{256}{3}$ \\ \\
Comparing the above area with the given area we get, the value of $a$ as  \\
\begin{align}
    \frac{32a^2}{3} = \frac{256}{3} \\ 
    a = 2\sqrt{2}
\end{align}
\begin{figure}[h!]
   \centering
   \includegraphics[width = 1\linewidth]{figs/fig.png}
   \caption{The parabola along with the line}
   \label{stemplot}
\end{figure}
\end{document}
