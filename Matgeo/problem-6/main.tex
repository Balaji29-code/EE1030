\let\negmedspace\undefined
\let\negthickspace\undefined
\documentclass[journal]{IEEEtran}
\usepackage[a5paper, margin=10mm, onecolumn]{geometry}
\usepackage{lmodern} % Ensure lmodern is loaded for pdflatex
\usepackage{tfrupee} % Include tfrupee package

\setlength{\headheight}{1cm} % Set the height of the header box
\setlength{\headsep}{0mm}     % Set the distance between the header box and the top of the text

\usepackage{gvv-book}
\usepackage{gvv}
\usepackage{cite}
\usepackage{amsmath,amssymb,amsfonts,amsthm}
\usepackage{algorithmic}
\usepackage{graphicx}
\usepackage{textcomp}
\usepackage{xcolor}
\usepackage{txfonts}
\usepackage{listings}
\usepackage{mathtools}
\usepackage{gensymb}
\usepackage{enumitem}
\usepackage{comment}
\usepackage[breaklinks=true]{hyperref}
\usepackage{tkz-euclide} 
\usepackage{listings}
\usepackage{gvv}                                        
\def\inputGnumericTable{}                                 
\usepackage[latin1]{inputenc}                                
\usepackage{color}                                            
\usepackage{array}                                            
\usepackage{longtable}                                       
\usepackage{calc}                                             
\usepackage{multirow}                                         
\usepackage{hhline}                                           
\usepackage{ifthen}                                           
\usepackage{lscape}
\begin{document}

\bibliographystyle{IEEEtran}
\vspace{3cm}

\title{4.4.2.14}
\author{EE24BTECH11010 - Balaji B}
% \maketitle
% \newpage
% \bigskip
{\let\newpage\relax\maketitle}
\textbf{Question:}\\
$x^2 + y^2 - 4x - 8y - 45 = 0$ \\
\textbf{Solution:}\\
\begin{table}[h!]    
  \centering
  
\begin{tabular}[12pt]{ |c|c|}
    \hline
    \textbf{Parameter} & \textbf{Description}\\ 
    \hline
    $\vec{x_1}$ & First intersection point\\
    \hline
    $\vec{x_2}$ & Second intersection point\\
    \hline
    $\vec{h}$ & Point on the given line\\
    \hline
    $\vec{m}$ & Direction vector of given line\\
    \hline
    $A$ & Area of the region\\
    \hline
    \end{tabular}


  \caption{Variables Used}
  \label{tab1-1.9-6}
  \end{table}\\
  The given circle can be expressed as \\
  \begin{align}
      \norm{\vec{x}}^2 +2\myvec{-2 && -4}\vec{x} - 45 = 0
  \end{align}
  where 
  \begin{align}
      \vec{u} = \myvec{-2 \\ -4}, f = -45 \\
      \implies \vec{c} = \myvec{2 \\ 4}, r = \sqrt{65}.
  \end{align}

    \begin{figure}[h!]
  \centering
  \includegraphics[width=1\linewidth]{figs/fig.png}
  \caption{Plot of the Circle}
   \label{stemplot}
\end{figure}

\end{document}
