\let\negmedspace\undefined
\let\negthickspace\undefined
\documentclass[journal]{IEEEtran}
\usepackage[a5paper, margin=10mm, onecolumn]{geometry}
%\usepackage{lmodern} % Ensure lmodern is loaded for pdflatex
\usepackage{tfrupee} % Include tfrupee package

\setlength{\headheight}{1cm} % Set the height of the header box
\setlength{\headsep}{0mm}     % Set the distance between the header box and the top of the text

\usepackage{gvv-book}
\usepackage{gvv}
\usepackage{cite}
\usepackage{amsmath,amssymb,amsfonts,amsthm}
\usepackage{algorithmic}
\usepackage{graphicx}
\usepackage{textcomp}
\usepackage{xcolor}
\usepackage{txfonts}
\usepackage{listings}
\usepackage{enumitem}
\usepackage{mathtools}
\usepackage{gensymb}
\usepackage{comment}
\usepackage[breaklinks=true]{hyperref}
\usepackage{tkz-euclide} 
\usepackage{listings}
% \usepackage{gvv}                                        
\def\inputGnumericTable{}                                 
\usepackage[latin1]{inputenc}                                
\usepackage{color}                                            
\usepackage{array}                                            
\usepackage{longtable}                                       
\usepackage{calc}                                             
\usepackage{multirow}                                         
\usepackage{hhline}                                           
\usepackage{ifthen}                                           
\usepackage{lscape}
\begin{document}

\bibliographystyle{IEEEtran}
\vspace{3cm}

\title{1-1.9-8}
\author{EE24BTECH11010 - BALAJI B}
% \maketitle
% \newpage
% \bigskip
{\let\newpage\relax\maketitle}

\renewcommand{\thefigure}{\theenumi}
\renewcommand{\thetable}{\theenumi}
\setlength{\intextsep}{10pt} % Space between text and floats


\numberwithin{equation}{enumi}
\numberwithin{figure}{enumi}
\renewcommand{\thetable}{\theenumi}

\textbf{Question :} \\ 
Construct a triangle $ABC$ in which $BC$ = 6cm, $CA$ = 5cm and $AB$ = 4cm. \\

\textbf{Answer :} \\

\begin{table}[h!]    
  \centering
  
\begin{tabular}[12pt]{ |c|c|}
    \hline
    \textbf{Parameter} & \textbf{Description}\\ 
    \hline
    $\vec{x_1}$ & First intersection point\\
    \hline
    $\vec{x_2}$ & Second intersection point\\
    \hline
    $\vec{h}$ & Point on the given line\\
    \hline
    $\vec{m}$ & Direction vector of given line\\
    \hline
    $A$ & Area of the region\\
    \hline
    \end{tabular}


  \caption{Variables Used}
  \label{tab1.9.19.1}
\end{table}

\begin{align}
    a &= BC = 7cm \\
    b &= AC \\
    c &= AB \\
    \therefore b + c = 13cm \\
    \since b^2 = a^2 +c^2 - 2ac\cosB \\
    \implies (13 - c)^2 = a^2 - 2ac\cosB \\
    \implies c = \frac{K^2-a^2}{2\brak{K-a\cos{B}}}
\end{align}
\end{document} 
