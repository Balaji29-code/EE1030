\let\negmedspace\undefined
\let\negthickspace\undefined
\documentclass[journal]{IEEEtran}
\usepackage[a5paper, margin=10mm, onecolumn]{geometry}
\usepackage{lmodern} % Ensure lmodern is loaded for pdflatex
\usepackage{tfrupee} % Include tfrupee package

\setlength{\headheight}{1cm} % Set the height of the header box
\setlength{\headsep}{0mm}     % Set the distance between the header box and the top of the text

\usepackage{gvv-book}
\usepackage{gvv}
\usepackage{cite}
\usepackage{amsmath,amssymb,amsfonts,amsthm}
\usepackage{algorithmic}
\usepackage{graphicx}
\usepackage{textcomp}
\usepackage{xcolor}
\usepackage{txfonts}
\usepackage{listings}
\usepackage{mathtools}
\usepackage{gensymb}
\usepackage{enumitem}
\usepackage{comment}
\usepackage[breaklinks=true]{hyperref}
\usepackage{tkz-euclide} 
\usepackage{listings}
\usepackage{gvv}                                        
\def\inputGnumericTable{}                                 
\usepackage[latin1]{inputenc}                                
\usepackage{color}                                            
\usepackage{array}                                            
\usepackage{longtable}                                       
\usepackage{calc}                                             
\usepackage{multirow}                                         
\usepackage{hhline}                                           
\usepackage{ifthen}                                           
\usepackage{lscape}
\begin{document}

\bibliographystyle{IEEEtran}
\vspace{3cm}

\title{31-01-2023- shift-2}
\author{EE24BTECH11010 - Balaji B}
% \maketitle
% \newpage
% \bigskip
{\let\newpage\relax\maketitle}
\begin{enumerate}
    \item Let the mean and standard deviation of marks of class $A$ of 100 students be respectively 40 and $\alpha \brak{>0}$, and the mean and standard deviation of marks of class $B$
 of $n$
 students be respectively $55$ and $30-\alpha$ 
. If the mean and variance of the marks of the combined class of $100 + n$
 students are respectively 50 and 350 , then the sum of variances of classes $A$
 and $B$
 is : \hfill (January-2023)
 \begin{enumerate}
     \begin{multicols}{2}
     \item 450
     \item 900
     \item 650 
     \item 500
     \end{multicols}
 \end{enumerate}
\item Let $\vec{a} = \hat{i} + 2\hat{j}+3\hat{k}, \vec{b} = \hat{i} - \hat{j}+ 2\hat{k}$
 and $\vec{c} = 5\hat{i} - 3\hat{j} + 3\hat{k}$
 be three vectors. If $\vec{r}$
 is a vector such that, $\vec{r} \times \vec{b} = \vec{c} \times \vec{b}$
 and $\vec{r} \cdot \vec{a} = 0$
, then $25\abs{\vec{r}}^2$
 is equal to : 
 
 \hfill(January-2023)
 \begin{enumerate}
     \begin{multicols}{2}
         \item 336
         \item 449
         \item 339
         \item 560
     \end{multicols}
 \end{enumerate}
 \item Let $H$ be the hyperbola, whose foci are $\brak{1 \pm \sqrt{2}, 0}$
 and eccentricity is $\sqrt{2}$ 
. Then the length of its latus rectum is : \hfill(January-2023)
\begin{enumerate}
    \begin{multicols}{2}
        \item $\frac{5}{2}$
        \item 3
        \item 2
        \item $\frac{3}{2}$
    \end{multicols}
\end{enumerate}
\item Let $\alpha > 0$. If $\int_0^{\alpha} \frac{x}{\sqrt{x + \alpha} - \sqrt{x}}dx = \frac{16 + 20\sqrt{2}}{15}$, then $\alpha$ is equal to: \hfill(January-2023)
\begin{enumerate}
    \begin{multicols}{2}
        \item 4
        \item 2
        \item $2\sqrt{2}$
        \item $\sqrt{2}$
    \end{multicols}
\end{enumerate}
\item The complex number $z = \frac{i - 1}{\cos{\frac{\pi}{3}} + i\sin{\frac{\pi}{3}}}$ is equal to:\hfill(January-2023)
\begin{enumerate}
    \begin{multicols}{2}
        \item $\sqrt{2}\brak{\cos{\frac{5\pi}{12}} + i\sin{\frac{5\pi}{12}}}$
         \item $\cos{\frac{\pi}{12}} - i\sin{\frac{\pi}{12}}$
          \item $\sqrt{2}\brak{\cos{\frac{\pi}{12}} + i\sin{\frac{\pi}{12}}}$
           \item $\sqrt{2} i \brak{\cos{\frac{5\pi}{12}} - i\sin{\frac{5\pi}{12}}}$
    \end{multicols}
\end{enumerate}
\item The coefficient of $x^{-6}$
, in the expansion of $\brak{\frac{4x}{5} + \frac{5}{2x^2}}^9$
, is

\hfill(January-2023)

\item Let the area of the region $\cbrak{\brak{x,y} : \abs{2x-1} \leq y \leq \abs{x^2 - x}, 0 \leq x \leq 1}$
 be $A$ . Then $\brak{6A + 11}^2$  
 is equal to 

 \hfill(January-2023)

 \item If $^{2n+1}P_{n-1} \text{ : } ^{2n-1}P_n = 11:21$, then $n^2 + n +15$ is equal to:

\hfill (January-2023)

\item If the constant term in the binomial expansion of 
$\brak{\frac{x^{\frac{5}{2}}}{2} - \frac{4}{x^l}}^9$ is $-84$ and the coefficient of $x^{-3l}$ 
 is $2^{\alpha}\beta$, where $\beta < 0$
 is an odd number, then $\abs{\alpha l - \beta}$
 is equal to

 \hfill(January-2023)

 \item Let $\vec{a}, \vec{b}, \vec{c}$
 be three vectors such that
$\abs{\vec{a}} = \sqrt{31}$, $4\abs{\vec{b}} = \abs{\vec{c}} = 2$
 and $2\brak{\vec{a} \times \vec{b}} = 3 \brak{\vec{c} \times \vec{a}}$
. If the angle between $\vec{b}$
 and $\vec{c}$
 is $\frac{2\pi}{3}$
, then $\brak{\frac{\vec{a} \times \vec{c}}{\vec{a}\cdot \vec{b}}}^2$ is equal to

\hfill (January-2023)

\item Let $S$ be the set of all $a \in \mathbb{N}$ such that the area of the triangle formed by the tangent at the point $P\brak{b,c}$, $b,c \in \mathbb{N}$, on the parabola $y^2 = 2ax $ and the lines $x = b, y = 0 $ is 16 unit$^2$, then $\sum_{a \in S} a $ is equal to 

\hfill (January-2023)

\item The sum $1^2 - 2\cdot3^2 + 3\cdot5^2 - 4\cdot7^2 + 5\cdot9^2 - \ldots + 15\cdot29^2$ is 

\hfill(January-2023)

\item Let $A$ be the event that the absolute difference between two randomly choosen real numbers in the sample space $\sbrak{0,60}$
 is less than or equal to $a$ . If $P\brak{A} = \frac{11}{36}$ , then $a$ 
 is equal to 

 \hfill (January-2023)

 \item Let $A = \sbrak{a_{ij}}, a_{ij} \in \mathbb{Z} \cap \sbrak{0,4}, 1 \leq i,j \leq 2$. 
The number of matrices $A$ such that the sum of all entries is a prime number $p \in \brak{2,13}$ is 

\hfill (January-2023)

\item Let $A$ be a $n \times n $ matrix such that $\abs{A} = 2$. If the determinant of the matrix $\operatorname{Adj}\brak{2\cdot \operatorname{Adj}\brak{2A^{-1}}}$ is $2^{84}$, then $n$ is equal to 

\hfill (January-2023)
\end{enumerate}
\end{document}
