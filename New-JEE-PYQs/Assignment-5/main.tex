\let\negmedspace\undefined
\let\negthickspace\undefined
\documentclass[journal]{IEEEtran}
\usepackage[a5paper, margin=10mm, onecolumn]{geometry}
\usepackage{lmodern} % Ensure lmodern is loaded for pdflatex
\usepackage{tfrupee} % Include tfrupee package

\setlength{\headheight}{1cm} % Set the height of the header box
\setlength{\headsep}{0mm}     % Set the distance between the header box and the top of the text

\usepackage{gvv-book}
\usepackage{gvv}
\usepackage{cite}
\usepackage{amsmath,amssymb,amsfonts,amsthm}
\usepackage{algorithmic}
\usepackage{graphicx}
\usepackage{textcomp}
\usepackage{xcolor}
\usepackage{txfonts}
\usepackage{listings}
\usepackage{mathtools}
\usepackage{gensymb}
\usepackage{enumitem}
\usepackage{comment}
\usepackage[breaklinks=true]{hyperref}
\usepackage{tkz-euclide} 
\usepackage{listings}
\usepackage{gvv}                                        
\def\inputGnumericTable{}                                 
\usepackage[latin1]{inputenc}                                
\usepackage{color}                                            
\usepackage{array}                                            
\usepackage{longtable}                                       
\usepackage{calc}                                             
\usepackage{multirow}                                         
\usepackage{hhline}                                           
\usepackage{ifthen}                                           
\usepackage{lscape}
\begin{document}

\bibliographystyle{IEEEtran}
\vspace{3cm}

\title{30-01-2024- shift-2}
\author{EE24BTECH11010 - Balaji B}
% \maketitle
% \newpage
% \bigskip
{\let\newpage\relax\maketitle}
\begin{enumerate}
    \item If $z$
 is a complex number, then the number of common roots of the equations $z^{1985} + z^{100} + 1 = 0 $ and $z^3 + 2z^2 + 2z + 1 = 0 $ 
, is equal to \hfill (January-2024)
\begin{enumerate}
    \begin{multicols}{2}
        \item 0
        \item 2
        \item 1
        \item 
    \end{multicols}
\end{enumerate}

  \item Suppose $2-p, p, 2 - \alpha, \alpha$
 are the coefficients of four consecutive terms in the expansion of $\brak{1 + x}^n$
. Then the value of $p^2 - \alpha^2 + 6\alpha + 2p$
 equals \hfill(January-2024)
 \begin{enumerate}
     \begin{multicols}{2}
        \item 8
        \item 4
        \item 6
        \item 10
     \end{multicols}
 \end{enumerate}
 \item If the domain of the function $f\brak{x} = \log_e\brak{\frac{2x +3}{4x^2 + x -3}} + \cos^{-1}\brak{\frac{2x-1}{x+2}} $ is $\brak{\alpha, \beta } $  , then the value of $5\beta - 4\alpha$
 is equal to
 \begin{enumerate}
     \begin{multicols}{2}
         \item 9
         \item 12
         \item 11
         \item 10
     \end{multicols}
 \end{enumerate}
 \item Let $f : \mathbb{R} \to \mathbb{R}$ be a function defined by $f\brak{x} = \frac{x}{\brak{1 + x^4} ^{\frac{1}{4}}}$ and $g\brak{x} = f\brak{f\brak{f\brak{f\brak{x}}}}$. Then, $18 \int_0^{\sqrt{2\sqrt{5}}} x^2 g\brak{x}dx$ is equal to \hfill(January-2024)
 \begin{enumerate}
     \begin{multicols}{2}
         \item 36
         \item 33
         \item 39
         \item 42
     \end{multicols}
 \end{enumerate}
 \item Let $R = \myvec{x & 0 & 0 \\ 0 & y & 0 \\ 0 & 0 & z}$ be a non-zero $3 \times 3$ matrix, where $x\sin{\theta} = y \sin{\brak{\theta + \frac{2\pi}{3}}} = z \sin{\brak{\theta + \frac{4\pi}{3}}} \ne 0, \theta \in \brak{0 , 2\pi}$. For a square matrix $M$, let trace $\brak{M}$ denote the sum of all diagonal entries of $M$. Then among the statements: \\ \\
 $\brak{\text{I}}$ Trace $\brak{R} = 0$\\
 $\brak{\text{II}}$ If trace $\brak{\operatorname{adj}\brak{\operatorname{adj\brak{R}}}} = 0$, then $R$ has exactly one non-zero entry.
 \hfill(January-2024)
 \begin{enumerate}  
 \begin{multicols}{2}
         \item Only (I) is true
         \item Only (II) is true
         \item Both (I) and (II) are true
         \item Neither (I) nor (II) is true
 \end{multicols}
 \end{enumerate}
 \item Let $Y = Y\brak{X}$
 be a curve lying in the first quadrant such that the area enclosed by the line $Y - y = Y^{\prime}\brak{X - x}$
 and the co-ordinate axes, where $\brak{x,y}$
 is any point on the curve, is always $ \frac{-y^2}{2Y^{\prime}\brak{x}} + 1, Y{\prime}\brak{x} \ne 0$. If $Y\brak{1} = 1$, then $12Y\brak{2}$
 equals 

 \hfill(January-2024)

 \item Let a line passing through the point $\brak{-1,2,3}$ intersect the lines $L_1 : \frac{x-1}{3} = \frac{y-2}{2} = \frac{z+1}{-2}$ 
 at $M\brak{\alpha, \beta, \gamma}$
 and $L_2 : \frac{x+2}{-3} + \frac{y-2}{-2} + \frac{z-1}{4}$ at $N\brak{a,b,c}$
. Then, the value of $\frac{\brak{\alpha+\beta + \gamma}^2}{\brak{a + b + c}^2}$
 equals

 \hfill(January-2024)

 \item Consider two circles $C_1 : x^2 + y^2 = 25$
 and $C_2 : \brak{x - \alpha} + y^2 = 16$
, where $\alpha \in \brak{5,9}$. Let the angle between the two radii (one to each circle) drawn from one of the intersection points of $C_1$
 and $C_2$
 be $\sin ^{-1} \brak{\frac{\sqrt{63}}{8}}$
. If the length of common chord of $C_1$
 and $C_2$
 is $\beta$
, then the value of $\brak{\alpha\beta}^2$
 equals

 \hfill(January-2024)
 \item Let $\alpha = \sum_{k = 0}^n \brak{\frac{\brak{^nC_k}^2}{k+1}}$ and $\beta = \sum_{k = 0}^{n-1} \brak{\frac{^nC_k\text{ } ^nC_{k+1}}{k+2}}$. If $5\alpha = 6\beta$, then $n$ equals 
 
 \hfil (January-2024)

 \item Let $S_n$
 be the sum to $n$-terms of an arithmetic progression $3,7,11$
, If $40 < \brak{\frac{6}{n(n+1)} \sum_{k = 1}^n S_k} < 42$
, then $n$
 equals 

 \hfill(January-2024)

 \item In an examination of Mathematics paper, there are 20 questions of equal marks and the question paper is divided into three sections : $A,B$
 and $C$
. A student is required to attempt total 15 questions taking at least 4 questions from each section. If section $A$
 has 8 questions, section $B$
 has 6 questions and section $C$
 has 6 questions, then the total number of ways a student can select 15 questions is 

 \hfill(January-2024)

 \item The number of symmetric relations defined on the set $\cbrak{1,2,3,4}$
 which are not reflexive is 

 \hfill(January-2024)

 \item The number of real solutions of the equation $x\brak{x^2 + 3\abs{x} + 5\abs{x-1} + 6 \abs{x-2}} = 0$
 is 

 \hfill (January-2024)

 \item The area of the region enclosed by the parabola $\brak{y-2}^2 = x -1 $
, the line $x - 2y + 4 = 0$
 and the positive coordinate axes is

 \hfill (January-2024)

 \item The variance $\sigma^2 $ of the data
 \begin{table}[h!]
     \centering
     
\begin{tabular}[12pt]{ |c|c|}
    \hline
    \textbf{Parameter} & \textbf{Description}\\ 
    \hline
    $\vec{x_1}$ & First intersection point\\
    \hline
    $\vec{x_2}$ & Second intersection point\\
    \hline
    $\vec{h}$ & Point on the given line\\
    \hline
    $\vec{m}$ & Direction vector of given line\\
    \hline
    $A$ & Area of the region\\
    \hline
    \end{tabular}


 \end{table} 
 
 Is

 \hfill (January-2024)
\end{enumerate}
\end{document}
