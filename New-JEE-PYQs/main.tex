\let\negmedspace\undefined
\let\negthickspace\undefined
\documentclass[journal]{IEEEtran}
\usepackage[a5paper, margin=10mm, onecolumn]{geometry}
\usepackage{lmodern} % Ensure lmodern is loaded for pdflatex
\usepackage{tfrupee} % Include tfrupee package

\setlength{\headheight}{1cm} % Set the height of the header box
\setlength{\headsep}{0mm}     % Set the distance between the header box and the top of the text

\usepackage{gvv-book}
\usepackage{gvv}
\usepackage{cite}
\usepackage{amsmath,amssymb,amsfonts,amsthm}
\usepackage{algorithmic}
\usepackage{graphicx}
\usepackage{textcomp}
\usepackage{xcolor}
\usepackage{txfonts}
\usepackage{listings}
\usepackage{mathtools}
\usepackage{gensymb}
\usepackage{enumitem}
\usepackage{comment}
\usepackage[breaklinks=true]{hyperref}
\usepackage{tkz-euclide} 
\usepackage{listings}
\usepackage{gvv}                                        
\def\inputGnumericTable{}                                 
\usepackage[latin1]{inputenc}                                
\usepackage{color}                                            
\usepackage{array}                                            
\usepackage{longtable}                                       
\usepackage{calc}                                             
\usepackage{multirow}                                         
\usepackage{hhline}                                           
\usepackage{ifthen}                                           
\usepackage{lscape}
\begin{document}

\bibliographystyle{IEEEtran}
\vspace{3cm}

\title{07-20-2021- shift-2}
\author{EE24BTECH11010 - Balaji B}
% \maketitle
% \newpage
% \bigskip
{\let\newpage\relax\maketitle}
 \begin{enumerate}
    \item For the natural numbers $m,n,$ if $(1-y)^m (1+y)^n = 1 + a_1y + a_2 y^2 + .... + a_{m+n} y^{m+n}$ and $a_1 = a_2 = 10 $, then the value of $(m+n)$ is equal to :
    \begin{enumerate}
        \item 88
        \item 64
        \item 100
        \item 80
    \end{enumerate}
    \item The value of $\tan\brak{2\tan^{-1}\frac{3}{5}+ \sin^{-1}\frac{5}{13}}$ is equal to :
    \begin{enumerate}
    \begin{multicols}{4}
        \item $\frac{-181}{69}$
        \item $\frac{220}{21}$
        \item $\frac{-291}{76}$
        \item $\frac{151}{63}$
    \end{multicols}
    \end{enumerate}
    \item  Let $r_1$ and $r_2$ be the radii of the largest and smallest circles, respectively, which pass through the point (-4,1) and having their centres on the circumference of the circle $x^2 + y^2 + 2x + 4y - 4 = 0$. If $\frac{r_1}{r_2} = a + b\sqrt{2} $, then $a+b$ is equal to:
    \begin{enumerate}
        \item 3
        \item 11
        \item 5
        \item 7 
    \end{enumerate}
    \item Consider the following three statements: \\
    $(A)$ If 3 + 3 = 7 then 4 + 3 = 8. \\ 
    $(B)$ If 5 + 3 = 8 then earth is flat.\\
    $(C)$ If both $(A)$ and $(B)$ are true then 5 + 6 = 17.\\
    Then, which of the following statements is correct ? 
    \begin{enumerate}
        \item (A) is false, but (B) and (C) are true 
        \item (A) and (C) are true while (B) is false
        \item (A) is true while (B) and (C) are false 
        \item (A) and (B) are false while (C) is true
    \end{enumerate}
    \item The lines $x = ay -1 = z 
- 2$ and $x = 3y -
 2 = bz -
 2, (ab \neq  
 0) $ are coplanar, if :
 \begin{enumerate}
     \item $b = 1 , a \in \textbf{R} - \{ 0 \}$
     \item $a = 1 , b \in \textbf{R} - \{ 0 \}$
     \item $a = 2, b = 2 $
     \item $ a =2 , b = 3$
 \end{enumerate}
   \item  If $\sbrak{x}$ denotes the greatest integer less than or equal to $x$, then the value of the integral $\int_{-\frac{\pi}{2}}^{\frac{\pi}{2}} \sbrak{\sbrak{x} - \sin{x}}dx$ is equal to :
    \begin{enumerate}
        \item $-\pi$
        \item $\pi$
        \item $0$
        \item 1
    \end{enumerate}
    \item  If the real part of the complex number $\brak{1 - \cos \theta + 2i \sin\theta}^{-1}$
  is $\frac{1}{5} \theta \in \brak{0 , \pi}$, then the value of the integral $\int_{0}^{\theta} \sin xdx$ is equal to:
  \begin{enumerate}
      \item 1
      \item 2
      \item  $-1$
      \item 0
  \end{enumerate}
  \item Let $f : \textbf{R} - \{ \frac{\alpha}{6} \} \to \textbf{R}$ be defined by 
 $f(x) = \frac{5x + 3}{6x - \alpha}.$ Then the value of $\alpha$
 for which $(fof)(x) = x$, for all $x = \textbf{R} - \{ \frac{\alpha}{6} \}$
, is :
\begin{enumerate}
    \item No such $\alpha$ exists
    \item 5 
    \item 8
    \item 6
\end{enumerate}
    \item If $f : \textbf{R} \to \textbf{R}$ is given by $f(x) = x + 1$
, then the value of  \[\lim_{n\to\infty} \frac{1}{n} \sbrak{f(0) + f\brak{\frac{5}{n}} + f \brak{\frac{10}{n}} + ..... + f\brak{\frac{5(n - 1)}{n}}},\]
\begin{enumerate}
\begin{multicols}{4}
    \item $\frac{3}{2}$
    \item $\frac{5}{2}$
    \item $\frac{1}{2}$
    \item $\frac{7}{2}$
    \end{multicols}
\end{enumerate}
    \item Let A, B and C be three events such that the probability that exactly one of A and B occurs is (1, 
 -$k$), the probability that exactly one of B and C occurs is (1, -2$k$), the probability that exactly one of C and A occurs is (1, -k) and the probability of all A, B and C occur simultaneously is $k^2$, where $0 < k < 1$. Then the probability that at least one of A, B and C occur is :
 \begin{enumerate}
     \item greater than $\frac{1}{8}$ but less than $\frac{1}{4}$
     \item greater than $\frac{1}{2}$
     \item greater than $\frac{1}{4}$ but less than $\frac{1}{2}$
     \item exactly equal to $\frac{1}{2}$
 \end{enumerate}
    \item The sum of all the local minimum values of the twice differentiable function $f : R \to
 R$ defined by  $f(x) = x^3 - 3x^2 - \frac{3f^{''}(2)}{2}x + f^{''}(1)$ is :
 \begin{enumerate}
     \item $-22$
     \item 5
     \item $-27$
     \item 0
 \end{enumerate}
  \item Let in a right angled triangle, the smallest angle be $\theta$. If a triangle formed by taking the reciprocal of its sides is also a right angled, then $\sin \theta$ is equal to:
 \begin{enumerate}
  \begin{multicols}{4}
      \item $\frac{\sqrt{5}+1}{4}$
      \item $\frac{\sqrt{5}-1}{2}$
      \item $\frac{\sqrt{2}-1}{2}$
      \item $\frac{\sqrt{5}-1}{4}$
  \end{multicols}
 \end{enumerate}
  \item Let $y = y(x)$ satisfies the equation $\frac{dy}{dx} - \abs{A} = 0$ for all $x > 0$, where $A$ = $\myvec{y & \sin x & 1 \\ 0 & -1 & 1 \\ 2 & 0 & \frac{1}{x}}$. If $y(\pi) = \pi + 2$, then the value of $y\brak{\frac{\pi}{2}}$ is:
 \begin{enumerate}
 \begin{multicols}{4}
     \item $\frac{\pi}{2} + \frac{4}{\pi}$
     \item $\frac{\pi}{2} - \frac{1}{\pi} $
     \item $\frac{3\pi}{2} - \frac{1}{\pi}$
     \item $\frac{\pi}{2} - \frac{4}{\pi} $
 \end{multicols}
 \end{enumerate}
  \item Consider the line L given by the equation $\frac{x-3}{2} = \frac{y-1}{1} = \frac{z-2}{1}$. Let Q be the mirror image of the point $(2,3,-1)$ with respect to L. Let a plane P be such that it passes through Q, and the line L is perpendicular to P. Then which of the following points is on the plane P?
 \begin{enumerate}
     \begin{multicols}{2}
     \item $(-1,1,2)$
     \item $(1,1,1)$
     \item $(1,1,2)$
     \item $(1,2,2)$
     \end{multicols}
 \end{enumerate}
 \item If the mean and variance of six observations $7, 10, 11, 15, a, b $ are 10 and $\frac{20}{3}$
 , respectively, then the value of $\abs{a -
 b}$ is equal to :
 \begin{enumerate}
     \item 9
     \item 11
     \item 7
     \item 1
 \end{enumerate}
 \end{enumerate}
\end{document}
