%iffalse
\let\negmedspace\undefined
\let\negthickspace\undefined
\documentclass[journal,12pt,twocolumn]{IEEEtran}
\usepackage{cite}
\usepackage{amsmath,amssymb,amsfonts,amsthm}
\usepackage{algorithmic}
\usepackage{graphicx}
\usepackage{textcomp}
\usepackage{xcolor}
\usepackage{txfonts}
\usepackage{listings}
\usepackage{enumitem}
\usepackage{mathtools}
\usepackage{gensymb}
\usepackage{comment}
\usepackage[breaklinks=true]{hyperref}
\usepackage{tkz-euclide} 
\usepackage{listings}
\usepackage{gvv}                                        
\def\inputGnumericTable{}                                 
\usepackage[latin1]{inputenc}                                
\usepackage{color}                                            
\usepackage{array}                                            
\usepackage{longtable}                                       
\usepackage{calc}                                             
\usepackage{multirow}                                         
\usepackage{hhline}                                           
\usepackage{ifthen}                                           
\usepackage{lscape}
\usepackage{tabularx}
\usepackage{array}
\usepackage{float}


\newtheorem{theorem}{Theorem}[section]
\newtheorem{problem}{Problem}
\newtheorem{proposition}{Proposition}[section]
\newtheorem{lemma}{Lemma}[section]
\newtheorem{corollary}[theorem]{Corollary}
\newtheorem{example}{Example}[section]
\newtheorem{definition}[problem]{Definition}
\newcommand{\BEQA}{\begin{eqnarray}}
\newcommand{\EEQA}{\end{eqnarray}}
\newcommand{\define}{\stackrel{\triangle}{=}}
\theoremstyle{remark}
\newtheorem{rem}{Remark}

% Marks the beginning of the document
\begin{document}
\bibliographystyle{IEEEtran}
\vspace{3cm}

\title{ASSIGNMENT 1}
\author{EE24BTECH11010 - BALAJI B}
\maketitle
\newpage
\bigskip

\renewcommand{\thefigure}{\theenumi}
\renewcommand{\thetable}{\theenumi}
\begin{enumerate}
    \item[2.] Five balls of different colours are to be placed in three boxes of different size. Each box can hold all five. In how many different ways can we place the balls so that no balls remain empty?\hfill $(1981- 4 \text{Marks})$
    \item[3.] m men and n women are seated in a row so that no two women sit together. If $m>n$, then show that the number of ways in which they can be seated is $dfrac{m!(m+1)!}{(m-n+1)!}$ \\

    \hfill$(1983-2\text{ Marks})$
    \item[4.] 7 relatives of a man comprises 4 ladies and 3 gentleman; his wife also has 7 relatives; 3 of them are ladies 4 gentleman. In how many ways they invite a dinner party of 3 ladies and 3 gentlemen so that there are 3 of man's relatives and 3 of wife's relatives?
    
    \hfill$(1985- 5 \text{ Marks})$
    \item[5.]A box contains two white balls, three black balls and four red balls. In how many ways can three balls be drawn from the box if at least one black ball is to be included in the draw?\hfill$(1986-2\text{ Marks})$  
    \item[6.] Eighteen guests have to be seated,half on each side of a long table. Four particular guests desire to sit on one particular side and three others on the other side. Determine the number of ways in which the sitting arrangements can be made.\hfill$(1991-4\text{ Marks})$
    \item[7.]A committee of 12 is to be formed from 9 women and 8 men. .In how many ways this can be done if at least five women have to be included in a committee? In how many of these committees \hfill $(1994- 4 \text{ Marks})$

    (a) The women are in majority?
    
    (b) The men are in majority?
    
    \item[8. ] Prove by permutation or otherwise $\frac{(n^2)!}{(n!)^n}$ is an integer $(n\in \mathbf{I}^+) $\hfill $(2004-2\text{ Marks})$  
     \item[9. ] If total number of runs scored in n matches is\\
     
     $\left(\frac{n+1}{4}\right)$ $(2^{n+1}-n-2)$ where $n>1$, and the run scored in the $k^{th}$ match are given by $k$. $2^{n+1-k}$, where$ 1 \leq k \leq n$. Find $n$ \hfill $(2005- 2 \text{ Marks})$
     
    \end{enumerate}
\newpage
\onecolumn
\begin{tabular}{p{1.5cm}p{10cm}}
   $\Large \textbf{F} $  &   \begin{Large}
   Match the Following 
   \end{Large}   
\end{tabular}
\noindent
\rule{1\textwidth}{1pt} % Width: 0.75\textwidth, Thickness: 1pt

\noindent
\begin{minipage}[t]{0.69\textwidth} 
    \textit{Each question contains statements given Column I in two columns, which have to be matched.The statements in Column-ll are labelled p, q, , s and t. Any given statement in Column-I can have correct matching with ONE OR MORE statement(s) in Column- 1. The appropriate bubbles corresponding to the answers to these questions have to be darkened as illustrated in the following example: }\\ \\
    \textit{If the correct matches are A-p, s and
bubbles will
t; B-q and r; C-p and q; and D-s then the correct darkening of look like the given.} 

    
\end{minipage}
\hfill
\begin{minipage}[t]{0.3\textwidth} 
    
\end{minipage}
\noindent
\rule{1\textwidth}{1pt} % Width: 0.75\textwidth, Thickness: 1pt

\begin{enumerate}
    \item[1. \hspace{0.5cm}] Consider all possible permutations of the ENDEANOEL. Match the Statements/ Expressions in
\textbf{Column I}with the Statements/ Expressions in \textbf{Column II} and indicate your answer by darkening
the appropriate bubbles in the 4x 4 matrix given in the ORS.\hfill$(2008)$\\ \begin{enumerate}
    \item[] \textbf{Column I} \hspace{10cm} \textbf{Column II}
    \vspace{0.8cm}
\end{enumerate}
\begin{tabular}{p{0.5cm}p{9cm}p{2.5cm}p{2cm}}
 (A) & The number of permutations containing the word ENDEA is & \hspace{2cm}(p) & 5! \\
    (B) & The number of permutations in which the letter E occurs in the first and the last positions is & \hspace{2cm}(q) & $2\times{5!}$\\
    (C)& The number of permutations in which none of the letters & \hspace{2cm}(r) &$7\times{5!}$\\
    (D) & The number of permutations in which the letters A, E, O occur only in odd positions is & \hspace{2cm}(s) & $21\times{5!}$
    
\end{tabular}
\vspace{0.5cm}
\item[2.\hspace{0.5cm}] in a high school, a committee has to be formed from a group of6 boys $M_1,M_2,M_3,M_4,M_5,M_6$ and 5 girls $G_1,G_2,G_3,G_4,G_5$.
\begin{enumerate}
    \item Let $\alpha_1$ be the total number of ways in which the committee can be formed such that the committee has 5 members, having exactly 3 boys and 2 girls. 
    \item Let $\alpha_2$ be the total number of ways in which the committee can be formed such that the committee has at least 2 members and having an equal number of boys and girls
    \item Let $\alpha_3$ be the total number of ways in which the committee can be formed such that the committee has 5 members, at least 2 of them being girls. 
    \item Let $\alpha_4$ be the total number of ways in which the committee can be formed such that the committee has 4 members, having at least 2 girls and such that both $M_1$ and $G_1$ are not $\mathbf{NOT} $ are not in the committee
\end{enumerate}



    
\end{enumerate}
\begin{tabular}{p{1cm}p{9cm}p{1cm}p{4cm}}
     &  $\mathbf{LIST-I}$ & & $\mathbf{LIST-II}$\\
    $\mathbf{P.}$  & The value of $\alpha_1$ is & $\mathbf{1.}$ & 136\\
    $\mathbf{Q.}$ & The value of $\alpha_2$ is & $\mathbf{2.}$&189\\
    $\mathbf{R.}$ & The value of $\alpha_3$ is & $\mathbf{3.}$&192\\
    $\mathbf{S.}$ & The value of $\alpha_4$ is & $\mathbf{4.}$&200\\
    & & $\mathbf{5.}$&381\\
    & & $\mathbf{6.}$&461\\
    
\end{tabular}\\

\hfill $(\mathbf{JEE} Adv. 2018)$\\
The correct option is:\\
\begin{tabular}{p{0.5cm}p{10cm}p{0.5cm}p{6cm}}
    (a) & $P \to4; Q\to6; R\to2; S\to 1$ & (b) & $P \to1; Q\to4; R\to2; S\to 3$ \\
     (c)& $P \to4; Q\to6; R\to5; S\to 2$ & (b) & $P \to4; Q\to2; R\to3; S\to 1$ 
\end{tabular}\\


\end{document}
